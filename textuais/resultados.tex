% ----------------------------------------------------------
% Capítulo de Resultados
% ----------------------------------------------------------
\chapter{Resultados}
\label{cap:resultados}
% ---

Em termos de gestão institucional, a plataforma trouxe uma modernização considerável para a Casa de Cultura, ao automatizar processos importantes como a gestão de editais e o acompanhamento da Escola de Artes. Com a seção de editais, foi possível integrar um sistema de inscrição online que facilita o processo de seleção e organização de eventos e concursos culturais, que antes era feita através de formulários do Google. Isso gera economia de tempo e recursos para os administradores e aumenta a transparência nas seleções, já que todas as informações ficam disponíveis de forma clara e acessível. Da mesma forma, a Escola de Artes se beneficiou de diversas funcionalidades que permitem o acompanhamento mais próximo dos alunos, com a informatização dos documentos acadêmicos, facilitando o acesso a informação entre alunos, professores e responsáveis.

Os testes realizados também reforçaram o bom desempenho da plataforma. Durante os testes de usabilidade, foi possível identificar que a navegação pela plataforma é intuitiva, com poucos ajustes necessários. Os usuários reportaram uma experiência positiva, tanto em termos de usabilidade quanto de layout visual, destacando a facilidade de localizar as principais funcionalidades e a clareza na exibição das informações.

Serão apresentados os resultados obtidos durante o processo de testes e validação da plataforma de artistas monlevadenses. Os testes foram realizados com base nos critérios de avaliação definidos previamente: funcionalidade, usabilidade, desempenho e segurança. A seguir, detalhamos os métodos aplicados para cada critério e os resultados observados.

\section{Critérios de Avaliação}

Funcionalidade: O principal objetivo foi verificar se as funcionalidades planejadas, como publicação de notícias, exposição dos artistas na "Vitrine", acompanhamento da Escola de Artes e divulgação de editais, estavam funcionando corretamente conforme o escopo do projeto.

Usabilidade: Avaliamos a facilidade de uso da plataforma, levando em consideração a experiência do usuário, navegabilidade e acessibilidade. Foi importante assegurar que a interface da plataforma fosse intuitiva, proporcionando uma experiência eficiente tanto para os artistas quanto para os administradores do sistema.

Desempenho: Medimos o tempo de resposta da plataforma sob diferentes cenários de uso, simulando múltiplos acessos simultâneos e a consulta de grandes volumes de dados. Verificamos a capacidade do sistema em lidar com essas demandas sem prejuízo à experiência do usuário.

Segurança: Avaliamos a integridade e a proteção dos dados inseridos na plataforma, garantindo que as informações dos usuários, especialmente dos artistas e administradores, estivessem protegidas contra acessos não autorizados.

\section{Métodos de Avaliação}
Testes de Usabilidade: Foram realizados testes com um grupo de usuários da Casa de Cultura, incluindo artistas e administradores. Esses testes tiveram como foco observar a interação dos usuários com as diferentes seções da plataforma, como o formulário de inscrição de artistas e o painel administrativo. Através de ferramentas de gravação de sessão e feedback direto, foram identificados pontos de melhoria, como o reposicionamento de alguns botões e a simplificação de etapas no processo de cadastro.

Testes de Desempenho: Utilizamos ferramentas automatizadas de teste de carga, simulando o acesso simultâneo de diversos usuários ao sistema, especialmente em áreas que demandam maior processamento, como a página de "Editais" e a "Vitrine de Artistas". O tempo médio de resposta do servidor foi inferior a 2 segundos, o que demonstra um desempenho eficiente para o volume esperado de acessos. No entanto, em testes de carga extrema, foi identificado um leve aumento no tempo de resposta, apontando áreas potenciais para otimização.

Análise de Logs: Foram analisados os logs gerados pela plataforma durante o período de testes, com o intuito de identificar possíveis erros ou comportamentos anômalos. Através dessa análise, foi possível verificar a consistência das operações de banco de dados, bem como a integridade das transações realizadas pelos administradores e usuários da plataforma. Pequenos erros de validação de dados foram corrigidos antes da finalização dos testes.

\section{Resultados Obtidos}
Os resultados dos testes indicam que a plataforma atende aos critérios de funcionalidade e usabilidade estabelecidos no início do projeto. As funcionalidades principais operam corretamente, e a interface foi bem avaliada pelos usuários quanto à sua facilidade de uso. Com relação ao desempenho, o sistema mostrou-se robusto, respondendo adequadamente à carga de usuários simulada, sendo necessário apenas pequenos ajustes para otimizar o tempo de resposta em cenários de alta demanda.

Quanto à segurança, os testes não revelaram vulnerabilidades críticas, e as medidas de proteção dos dados se mostraram eficazes. Foram implementadas rotinas de backup automáticas e mecanismos de autenticação robustos para garantir a integridade das informações.

Em resumo, a plataforma da Casa de Cultura de João Monlevade se mostrou eficiente nos testes realizados, com bom desempenho, usabilidade adequada e segurança confiável, estando apta a ser utilizada pelos artistas locais e administradores da instituição.