% ----------------------------------------------------------
% Capitulo de revisão de literatura
% ----------------------------------------------------------
\chapter{Revisão bibliográfica}
\label{cap:revisao}
% ----------------------------------------------------------

    Gestão cultural e projetos culturais:
        Bianchini, F. (2004). Gestão de projetos culturais. Porto Alegre: Bookman.
        Teixeira, F., & Rocha, S. (2014). Gestão cultural: reflexões, conceitos e estudos de caso. Porto: Afrontamento.

    Plataformas digitais e tecnologia na cultura:
        Monteiro, C., & Cardoso, G. (2015). Plataformas digitais: impactos e desafios para a cultura. Comunicação e Sociedade, 27, 197-214.
        Valente, J. A. (2019). Tecnologia e Cultura: da Mediação à Participação. Educação & Sociedade, 40, e019009.

    Empreendedorismo e economia criativa:
        Florida, R. (2002). The Rise of the Creative Class: And How It's Transforming Work, Leisure, Community and Everyday Life. New York: Basic Books.
        Howkins, J. (2013). The Creative Economy: How People Make Money from Ideas. London: Penguin.

    Marketing e promoção cultural:
        Kotler, P., & Scheff, J. (1995). Standing room only: Strategies for marketing the performing arts. Boston, MA: Harvard Business Press.
        Do Valle, R. (2004). Marketing cultural: a arte de conquistar patrocínios. São Paulo: Atlas.
