% ----------------------------------------------------------
% Capitulo de revisão de literatura
% ----------------------------------------------------------
\chapter{Revisão bibliográfica}
\label{cap:revisao}
% ----------------------------------------------------------
Este capítulo tem como objetivo fornecer o contexto teórico para o desenvolvimento da plataforma de artistas da Casa de Cultura de João Monlevade. Serão apresentados conceitos fundamentais sobre as tecnologias envolvidas na criação de sistemas web, destacando questões de usabilidade e segurança aplicadas no projeto. Também serão analisados trabalhos relacionados ao tema.

Nesta primeira seção, será apresentada a Casa de Cultura de João Monlevade, abordando sua história, papel no desenvolvimento cultural da cidade e sua importância como centro de promoção das expressões artísticas locais. Além disso, serão destacadas suas atividades e contribuições ao longo dos anos, evidenciando sua relevância tanto na preservação da memória cultural quanto na formação de novos talentos e na integração com a comunidade monlevadense.


\section{Introdução à Casa de Cultura de João Monlevade}
%Confirmar dados com a casa de cultura 
A Casa de Cultura de João Monlevade desempenha um papel importante no desenvolvimento social e na promoção da cultura local. A instituição foi fundada em 1948 por uma iniciativa de um grupo de intelectuais e artistas locais que perceberam a necessidade de um local dedicado à apreciação das expressões culturais monlevadenses. A Casa de Cultura tem sido um local onde as pessoas se reúnem e celebram a cultura desde então, e ela desempenha um papel importante na vida cultural da cidade.

Ao longo dos anos, a Casa de Cultura tem se destacado por sua diversidade de atividades e qualidade. Apresentações musicais, escola de artes,oficinas, dança e literatura estão entre as atividades da instituição. Além de promover a cultura local, essas atividades visam incentivar a produção artística e intelectual da região, promovendo o debate cultural na comunidade.

A Casa de Cultura tem um público diversificado, desde crianças até idosos. Eles sempre estão procurando oferecer programações e atividades que atendam às diferentes faixas etárias e interesses. Além disso, a organização mantém fortes vínculos com as escolas e com os grupos comunitários, promovendo de forma integrada a educação e a cultura.

O significado da Casa de Cultura de João Monlevade vai além da área cultural. A organização ajuda a construir uma consciência histórica e cultural entre os habitantes locais, promovendo a identidade e a memória da cidade. Além disso, a Casa de Cultura contribui para o enriquecimento da vida cultural da cidade, desempenhando um papel importante na formação de novos talentos e na promoção da diversidade cultural.

Atualmente, a Casa de Cultura de João Monlevade está presente em diversas redes sociais, como por exemplo o Instagram e o Facebook. Através de suas páginas nessas redes, os funcionários da Casa de Cultura publicam sobre eventos próximos e futuros organizados por eles, além de divulgar vagas para os cursos da escola de artes, presente dentro da sede. Essa interação mais próxima com a população se tornou cada vez mais importante, tendo em vista a facilidade de comunicação com o publico alvo através dessas ferramentas.

Assim, a Casa de Cultura de João Monlevade é mais do que apenas um promotor de eventos culturais. Ela se consolida como um símbolo da identidade e da história da cidade, atuando também como um centro de irradiação cultural que enriquece a vida dos habitantes e promove a inclusão social através da cultura. Seus esforços contínuos para preservar e divulgar a cultura local são inestimáveis, tornando a instituição um patrimônio cultural e social de grande valor para a comunidade monlevadense.


\section{A relevância da internet e dos sites web para a cultura}

Nesse contexto, o papel da internet se torna ainda mais relevante. A revolução digital expandiu as fronteiras culturais, permitindo que instituições como a Casa de Cultura alcancem um público muito além das limitações físicas. Hoje, não há dúvida de quão importante a internet e os sites web se tornaram para a cultura. Com a crescente adoção da internet como meio de comunicação e divulgação cultural, as organizações culturais encontram uma ferramenta eficaz para atingir um público mais amplo e diversificado, conectando diferentes comunidades e promovendo o acesso à cultura de forma democrática. Essa dinâmica reflete a importância das políticas públicas culturais no Brasil, como os 'Pontos de Cultura', que promovem o diálogo intercultural e o acesso à diversidade cultural por meio da internet \cite{costa2011}.

Uma das principais vantagens de ter um site web para uma instituição cultural é aumentar sua visibilidade e alcance. Atualmente, sem um grande investimento em publicidade e marketing, uma instituição cultural poderia alcançar não somente o público local, mas também aqueles de outras regiões e até mesmo de outros países.

Além disso, o público interessado pode acessar informações por meio de um site. Ele pode fornecer informações sobre a história, missão, projetos e atividades da instituição. Também pode fornecer informações práticas, como horários de funcionamento, formas de contato e rotas de localização. Isso facilita a interação do público com a instituição e sua participação em suas atividades.

A capacidade de promover eventos e atividades culturais de forma mais eficaz é outra vantagem de ter um site web. Ao divulgar programações, exposições, espetáculos, cursos e workshops no site, é possível alcançar um público maior e garantir que mais pessoas participem das atividades da instituição. No caso de Teresina-PI, por exemplo, os eventos culturais contribuem significativamente para a promoção turística e geração de empregos, embora demandem uma divulgação mais eficaz para maximizar seu impacto \cite{gomes2009}.

Como resultado, a internet e os sites web são hoje uma ferramenta vital para as instituições culturais, permitindo não apenas um maior alcance e visibilidade, mas também uma maior interação e participação do público. Uma instituição cultural desenvolvendo um site web está contribuindo para a promoção da cultura e o fortalecimento da identidade cultural da comunidade.

A digitalização da cultura e a presença online das instituições culturais também permitem a preservação e disseminação do patrimônio cultural de maneira inédita. As redes sociais desempenham um papel importante na preservação dos patrimônios imateriais, ampliando a participação da sociedade nos processos de patrimonialização cultural digital \cite{ramires2019}. Essa disponibilidade perpetua o conhecimento cultural, tornando-o acessível para futuras gerações, independentemente das barreiras geográficas.

Além da disseminação de informações, os sites web oferecem um espaço para a experimentação e inovação cultural. Plataformas online podem ser utilizadas para criar exposições virtuais, apresentações interativas e outras formas de arte digital, expandindo as possibilidades de expressão artística. Estudos mostram que a inovação tecnológica e digital tem facilitado a participação do público e aberto novos caminhos para a expressão cultural \cite{lu2023}.

Outro aspecto importante é o papel dos sites web na educação cultural. Instituições culturais podem oferecer cursos online, workshops e palestras, acessíveis a qualquer pessoa com conexão à internet. Esse tipo de oferta educativa amplia o impacto social das instituições culturais, permitindo que mais pessoas tenham acesso a conhecimentos especializados e a oportunidades de desenvolvimento pessoal e profissional. Estudos indicam que a cultura digital, por meio de plataformas online, facilita o acesso global à educação cultural, personalizando as abordagens de ensino através de tecnologias digitais \cite{xiong2019}.

Finalmente, os sites web permitem uma comunicação bidirecional entre as instituições culturais e seu público. Por meio de comentários, fóruns e redes sociais integradas, os usuários podem interagir diretamente com as instituições, expressar suas opiniões e participar de discussões sobre temas culturais. Essa interação promove um senso de comunidade e pertencimento, além de proporcionar às instituições um feedback valioso para melhorar suas ofertas e serviços.

Em conclusão, a internet e os sites web transformaram a maneira como a cultura é compartilhada, promovida e consumida. Eles oferecem inúmeras vantagens para as instituições culturais, desde o aumento da visibilidade e alcance até a preservação do patrimônio cultural e a inovação artística. Ao desenvolver uma presença online robusta, as instituições culturais não apenas ampliam seu impacto, mas também fortalecem a identidade cultural e a coesão social das comunidades que servem.

\section{Tecnologias para desenvolvimento de sites web}

As principais tecnologias utilizadas para o desenvolvimento web são bem difundidas no meio da programação. Há, primeiramente, no nível básico, o \ac{HTML}, que seria similar à uma folha de papel onde é possível estruturar o site. Para configurar a parte gráfica a ajustar informações de fonte, tamanho e cor, usa-se o \ac{CSS}, como se a folha de papel pudesse ser colorida. E caso seja necessário que essa folha de papel tenha alguma animação, ou uma automação de algum tipo, usa-se o JavaScript para dar vida à página \cite{W3C}.

Apesar da possibilidade de desenvolver diversas páginas utilizando apenas as três tecnologias fundamentais - \ac{HTML}, \ac{CSS} e JavaScript, a prática demonstra que, em determinado momento, surgem limitações que podem dificultar o processo de desenvolvimento. Assim, para contornar esses desafios, foram criados frameworks e bibliotecas que visam facilitar e agilizar o desenvolvimento web. Entre essas ferramentas, destaca-se o React, uma biblioteca JavaScript amplamente utilizada para a construção de interfaces de usuário, especialmente em \ac{SPAs} como exemplificado por \citeonline{mukhiya2018architectural}. Outro exemplo significativo é o Angular, um framework desenvolvido pelo Google, que permite a criação de aplicativos web dinâmicos e responsivos, otimizando a experiência do usuário e melhorando a interação com as interfaces \cite{green2013angularjs}.

Quando se precisa de um framework ainda mais robusto, que auxilie na criação de sites no estilo blog, ou que armazenaram dados, notícias ou similares, pode-se utilizar sistemas \ac{CMS}. Esses sistemas facilitam ainda mais a criação e manutencao de sites com varias funcionalidades, ajudando na utilizacao de banco de dados, criação de conteúdo e ate mesmo na \ac{SEO}. Dentre os principais atualmente temos o Wordpress, o \ac{CMS} mais popular, usado para criar e gerenciar sites e blogs com facilidade. Ele oferece uma vasta gama de plugins e temas, facilitando a personalização e a adição de funcionalidades ao site. O Joomla \cite{Joomla} é um \ac{CMS} de código aberto que permite a criação de sites poderosos e complexos que oferece flexibilidade e uma comunidade ativa, tornando-o uma escolha robusta para sites de médio a grande porte. O Drupal \cite{Drupal}, um \ac{CMS} altamente flexível e escalável, é ideal para sites que requerem uma personalização intensa, sendo também conhecido por sua segurança e capacidade de gerenciar grandes volumes de conteúdo \cite{canavan2011cms}.

Dado o anterior, acredita-se ser de grande importância a utilização de um sistema \ac{CMS}. Para definir qual das tecnologias seria mais adequada, é necessário atentar-se às funcionalidades desse sistema e qual objetivo ele deve cumprir. Tendo em vista que o site para a Casa de Cultura de João Monlevade teria dois objetivos principais, sendo esses servir como meio para divulgar noticias e editais da Casa de Cultura e ser também um sistema de gestão para a Escola de Artes, optou-se um \ac{CMS} que possibilitasse essa multitarefa.

O Wagtail CRX é um sistema de gerenciamento de conteúdo baseado no framework Django e no \ac{CMS} Wagtail. Ele foi projetado pela CodeRed LLC para facilitar a criação e a gestão de sites empresariais, com foco em flexibilidade, desempenho e facilidade de uso, extendendo as funcionalidades já presentes nesses dois sistemas \cite{WagtailCRX}. o Django é um dos frameworks mais robustos e escaláveis para desenvolvimento web, conhecido por sua segurança e capacidade de lidar com grandes volumes de dados e tráfego. Já o Wagtail é um \ac{CMS} flexível e poderoso, desenvolvido sobre o Django, que oferece uma interface de administração amigável e recursos avançados de gestão de conteúdo \cite{Wagtail}.

Outra funcionalidade presente no Wagtail é sua avançada ferramenta para gestão de conteúdo. A flexibilidade na criação de páginas, permite a confecção de páginas personalizadas com componentes reutilizáveis, facilitando a construção de layouts detalhados sem a necessidade de códigos complexos ou repetitivos. O sistema também auxilia na publicação de notícias, oferecendo ferramentas integradas para a publicação e gerenciamento dos conteúdos, com recursos de agendamento e controle de versões.

Além de uma interface amigável para a criação de novas publicações como artigos e editais, o sistema possibilita a utilização de outra interface que poderá ser acessada por administradores do sistema, o Django admin. Esta vem por padrão quando se utiliza o Django e oferece facilidade na modificação de informações, além de acesso direto aos dados guardadas no banco de dados do site, o que será importante para as funcionalidades da Escola de Artes.

Outra vantagem significativa do Wagtail CRX é a sua robusta estrutura de segurança. Desenvolvido sobre o Django, que é conhecido por suas práticas de segurança integradas, o sistema oferece proteção contra ataques comuns, como \textit{SQL injection}, \ac{XSS} e \ac{CSRF}. Isso é particularmente importante para a Casa de Cultura de João Monlevade, que pode lidar com dados sensíveis dos usuários \cite{DjangoCSRF}.

A escalabilidade do Wagtail CRX também é um ponto forte. Conforme a Casa de Cultura de João Monlevade cresce e expande suas atividades, o \ac{CMS} pode acompanhar esse crescimento sem comprometer o desempenho. O Django, com seu suporte a grandes volumes de dados e tráfego, assegura que o site permanecerá rápido e responsivo, mesmo com um aumento significativo no número de visitantes e conteúdos publicados \cite{ghimire2020comparative}.

A facilidade de uso do Wagtail, que é a base do Wagtail CRX, proporciona uma experiência administrativa intuitiva. A interface de arrastar e soltar, juntamente com a capacidade de pré-visualizar as alterações antes de publicá-las, facilita o trabalho dos administradores do site, que podem não ter habilidades técnicas avançadas. Isso permite que a equipe da Casa de Cultura de João Monlevade se concentre mais na criação e gestão de conteúdo de qualidade, ao invés de se preocupar com os aspectos técnicos da manutenção do site.

Além disso, o Wagtail CRX suporta práticas de \ac{SEO} avançadas, essenciais para aumentar a visibilidade online da Casa de Cultura de João Monlevade. Ferramentas integradas ajudam a otimizar o conteúdo para motores de busca, melhorando o ranking do site nos resultados de pesquisa e atraindo mais visitantes. Com uma estratégia de \ac{SEO} bem implementada, a Casa de Cultura pode alcançar um público mais amplo, promovendo suas atividades e eventos de maneira mais eficaz \cite{WagtailCRXSEO}.

A escolha do Wagtail CRX também se justifica pela possibilidade de personalização e extensão das suas funcionalidades. A Casa de Cultura de João Monlevade pode, por exemplo, desenvolver módulos específicos para atender a necessidades particulares, como a gestão de inscrições para cursos e workshops, acompanhamento de presença dos alunos, e avaliação de performances artísticas. Essa capacidade de personalização garante que o sistema possa evoluir junto com as demandas específicas da instituição.

A utilização do Wagtail CRX também contribui para a sustentabilidade da Casa de Cultura de João Monlevade. Um sistema eficiente de gerenciamento de conteúdo reduz a necessidade de recursos físicos, como papel para folhetos e cartazes, substituindo-os por versões digitais acessíveis online. Isso não apenas diminui os custos operacionais, mas também apoia práticas ambientalmente responsáveis.

A interface do usuário do Wagtail, intuitiva e amigável, facilita a capacitação da equipe da Casa de Cultura. Com um treinamento básico, os administradores do site podem rapidamente aprender a criar, editar e publicar conteúdos, garantindo que a plataforma seja utilizada de maneira eficiente. A facilidade de uso incentiva a equipe a atualizar regularmente o site, mantendo-o sempre relevante e atualizado para os visitantes.

Em termos de design, o Wagtail CRX permite a implementação de um site visualmente atraente e alinhado com a identidade visual da Casa de Cultura de João Monlevade. A flexibilidade na personalização do layout e a disponibilidade de templates pré-desenvolvidos proporcionam um design profissional sem a necessidade de investimentos significativos em desenvolvimento gráfico. Isso é fundamental para transmitir a imagem e os valores da instituição de maneira coerente e atraente.

Por fim, o suporte a múltiplos idiomas é uma funcionalidade relevante do Wagtail CRX. Considerando a diversidade cultural e o potencial interesse de visitantes internacionais, a possibilidade de oferecer conteúdo em diferentes idiomas amplia o alcance do site e facilita o acesso de um público global. Essa funcionalidade é particularmente importante para a promoção de eventos e exposições que possam atrair turistas e estudiosos de outros países.

Em síntese, a escolha do Wagtail CRX para o desenvolvimento do site da Casa de Cultura de João Monlevade é justificada por uma série de vantagens, que vão desde sua flexibilidade e segurança até sua escalabilidade e capacidade de personalização. Além disso, o sistema oferece ferramentas robustas para otimização de \ac{SEO} e suporte ao marketing digital, o que potencializa a visibilidade da plataforma. Outros aspectos relevantes a serem considerados são a usabilidade e acessibilidade, temos estes que serão abordados à continuação. Ambos são essenciais para garantir que o site atenda a todos os públicos de maneira eficiente e inclusiva.

\section{Usabilidade e acessibilidade de sites web}

A usabilidade e a acessibilidade são componentes essenciais para o sucesso de qualquer plataforma digital. A usabilidade diz respeito à facilidade com que os usuários conseguem navegar e realizar suas tarefas em um site, envolvendo aspectos como eficiência, eficácia e satisfação. Já a acessibilidade se refere à capacidade do site de ser utilizado por pessoas com diversas deficiências, garantindo que ninguém seja excluído da experiência digital. Para que um site seja verdadeiramente inclusivo e funcional, é fundamental seguir diretrizes que promovam esses dois elementos, assegurando que todos os públicos, independentemente de suas habilidades, possam interagir com a plataforma de maneira fluida e satisfatória \cite{rodrigues2008acessibilidade}.

Primeiramente, a simplicidade e clareza da interface são fundamentais para a usabilidade. Uma interface bem projetada deve ser intuitiva e fácil de entender, mesmo para usuários que a estão acessando pela primeira vez. Isso inclui a organização lógica dos elementos, a utilização de uma linguagem clara e a minimização da complexidade. A adoção de uma abordagem minimalista, onde apenas os elementos essenciais são destacados, pode ajudar a evitar a sobrecarga de informações e tornar a navegação mais fluida.

Além da simplicidade, a navegação intuitiva é um aspecto vital para a usabilidade. Os menus e as opções de navegação devem ser organizados de forma coerente, permitindo que os usuários encontrem facilmente o que procuram. Elementos de navegação como barras de menu, botões e links devem ser visíveis e consistentes em todas as páginas do site. Adotar padrões conhecidos e evitar mudanças drásticas na estrutura de navegação pode melhorar significativamente a experiência do usuário \cite{islam2016towards}.

A acessibilidade vai além da usabilidade, garantindo que o conteúdo do site esteja disponível e utilizável por pessoas com deficiências visuais, auditivas e motoras. Para alcançar esse objetivo, é necessário seguir diretrizes específicas, como as estabelecidas pela \ac{WCAG} \cite{wcag2020}. Essas diretrizes fornecem recomendações detalhadas para tornar o conteúdo web mais acessível.

Para pessoas com deficiências visuais, é importante que o site seja compatível com leitores de tela. Isso inclui o uso adequado de tags alt para descrever imagens, a estruturação correta do \ac{HTML} para que os leitores de tela possam interpretar a página de maneira lógica e a garantia de contraste adequado entre o texto e o fundo para facilitar a leitura. Além disso, oferecer opções de aumentar o tamanho da fonte pode ser benéfico para usuários com baixa visão.

Para tornar o conteúdo acessível a pessoas com deficiências auditivas, é essencial fornecer alternativas textuais para qualquer conteúdo de áudio ou vídeo. Legendas, transcrições e descrições de áudio são recursos importantes que garantem que essas informações possam ser acessadas por todos. Em vídeos, as legendas sincronizadas e as transcrições ajudam a transmitir a mensagem de maneira eficaz.

A acessibilidade para pessoas com deficiências motoras pode ser melhorada garantindo que o site possa ser navegado usando apenas o teclado, sem a necessidade de um mouse. Os elementos clicáveis devem ser suficientemente grandes e espaçados para facilitar o uso por pessoas com mobilidade limitada. Além disso, implementar atalhos de teclado pode acelerar a navegação para usuários que dependem desses dispositivos.

Para garantir que a Casa de Cultura de João Monlevade ofereça um site inclusivo, é vital incorporar as melhores práticas de usabilidade e acessibilidade desde o início do processo de desenvolvimento. Uma abordagem proativa à acessibilidade não apenas beneficia os usuários com deficiências, mas também melhora a experiência geral para todos os visitantes do site.

Para promover a inclusão digital, é importante também considerar a acessibilidade em dispositivos móveis. Com o crescente uso de smartphones e tablets para acessar a internet, o site da Casa de Cultura de João Monlevade deve ser responsivo e oferecer uma experiência de usuário consistente em todas as plataformas. Garantir que o design seja \textit{mobile-first}, ou seja, otimizado primeiro para dispositivos móveis e depois adaptado para desktops, pode ser uma estratégia eficaz para alcançar essa meta.

Assim, é essencial que a acessibilidade e a usabilidade sejam vistas como uma responsabilidade contínua e colaborativa. Manter um diálogo aberto com a comunidade de usuários, incluindo aqueles com deficiências, e estar disposto a adaptar e melhorar o site com base em suas necessidades e feedbacks, é a chave para criar um ambiente digital verdadeiramente inclusivo. A construção do site, ao adotar uma abordagem centrada no usuário para a acessibilidade, não apenas cumpre suas obrigações legais e éticas, mas também reforça seu compromisso com a inclusão e a equidade.


\section{Segurança da informação em sites web}

No entanto, além de garantir que o site seja acessível e fácil de usar, outro aspecto fundamental para sua eficácia é a segurança da informação. Uma plataforma que atende a um público diverso também precisa assegurar que os dados de todos os usuários sejam protegidos contra ameaças cibernéticas. A proteção da informação em sites da web é uma questão crucial, especialmente para instituições como a Casa de Cultura de João Monlevade, que lida com dados sensíveis de usuários e informações essenciais para suas atividades. Os principais desafios em termos de segurança incluem as ameaças de hackers, malware e phishing, todos capazes de causar danos significativos, desde a interrupção dos serviços até o comprometimento dos dados pessoais dos usuários.

Os ataques cibernéticos representam uma das maiores ameaças à segurança dos sites da web. Esses ataques podem se manifestar de várias formas, incluindo invasões para roubo de informações, \textit{defacement} (alteração maliciosa do conteúdo do site) e ataques de \ac{DDoS}, que podem tornar o site inacessível aos usuários. Para mitigar esses riscos, é vital adotar medidas robustas de segurança, como firewalls, sistemas de detecção de intrusos e a configuração adequada dos servidores \cite{li2021comprehensive}.

Outras ameaças relevantes são o \textit{malware}, programa malicioso criado para danificar, explorar ou desativar sistemas computacionais, \textit{phishing}, que é uma técnica de engenharia social usada para enganar os usuários e obter informações confidenciais, \ac{XSS}, aonde scripts modificados são injetados em sites confiáveis e \textit{SQL Injection}, uma técnica de injeção de código \ac{SQL} que pode expôr informações sensíveis à atores mal intencionados. Para manter o site seguro contra malwares, é fundamental manter o software sempre atualizado, fazer verificações de segurança frequentes e garantir que todos os plugins e temas sejam baixados de fontes confiáveis e atualizados regularmente considerando que as atualizações de software frequentemente incluem patches de segurança que corrigem falhas recentemente descobertas, tornando o site mais seguro contra ataques.

Para garantir a segurança do site da Casa de Cultura de João Monlevade, a implementação de protocolos de segurança é essencial. Isso inclui o uso de \ac{HTTPS} para criptografar a comunicação entre o navegador do usuário e o servidor, protegendo assim as informações transmitidas contra interceptação. 

A atualização constante do software é outra medida crítica para a segurança do site. Manter o sistema operacional, servidores, \ac{CMS} e todos os plugins e temas atualizados é vital para proteger contra vulnerabilidades conhecidas que podem ser exploradas por hackers. 

A proteção contra ataques cibernéticos também envolve a realização de backups regulares dos dados do site. Ter backups atualizados garante que, em caso de um ataque bem-sucedido ou falha do sistema, o site possa ser restaurado rapidamente com a perda mínima de dados. Além disso, é recomendável implementar soluções de monitoramento e resposta a incidentes, que podem detectar atividades suspeitas e responder rapidamente para mitigar os danos \cite{baykara2018overview}.

A conscientização e o treinamento dos funcionários também desempenham um papel crucial na segurança da informação. Os funcionários devem ser treinados para reconhecer e responder a ameaças cibernéticas, como e-mails de phishing e outras tentativas de engenharia social. Programas de treinamento contínuo garantem que todos na organização estejam atualizados sobre as melhores práticas de segurança e saibam como proteger as informações sensíveis do site e dos usuários.

Outro aspecto importante é a gestão de acessos e permissões. O princípio do menor privilégio deve ser aplicado, garantindo que cada usuário tenha apenas o acesso necessário para realizar suas tarefas \cite{schneider2003least}. Contas com privilégios administrativos devem ser limitadas e monitoradas rigorosamente para evitar abuso de acesso. Além disso, é importante desativar ou remover contas de usuários que não são mais necessárias para minimizar o risco de uso indevido.

Essas práticas não apenas protegem contra as ameaças cibernéticas, mas também garantem a continuidade das operações e a integridade das informações. A segurança deve ser vista como um componente essencial do desenvolvimento e manutenção do site, integrando-se a todas as fases do ciclo de vida do software. Através de um compromisso contínuo com a segurança da informação, a Casa de Cultura de João Monlevade pode cumprir sua missão de promover a cultura e servir a comunidade de maneira segura e eficiente.

\section{Trabalhos Relacionados}

Além de proteger os dados e assegurar a integridade do sistema, o desenvolvimento de uma plataforma cultural também depende da análise de outras iniciativas semelhantes. Nesse contexto, é crucial observar como outras Casas de Cultura, tanto no Brasil quanto no exterior, abordam suas plataformas digitais. A análise de trabalhos relacionados é, portanto, um passo importante para compreender as melhores práticas adotadas por essas instituições \cite{boud2003learning}. Este capítulo explora exemplos de sites web de Casas de Cultura, focando em suas funcionalidades, design, usabilidade e na qualidade do conteúdo disponibilizado. A identificação dessas boas práticas oferece uma base sólida para que o site da Casa de Cultura de João Monlevade possa ser desenvolvido de forma eficiente e alinhada às necessidades da comunidade.

Para compreender melhor as melhores práticas na criação de sites para Casas de Cultura, foram analisados vários exemplos, tanto nacionais quanto internacionais. A seleção incluiu instituições com características semelhantes à Casa de Cultura de João Monlevade, permitindo uma comparação direta e a extração de elementos aplicáveis ao contexto local.

A Fundação Clovis Salgado, localizada em Belo Horizonte, possui um site que se destaca pela clareza e acessibilidade das informações. Entre suas funcionalidades, incluem-se a divulgação de eventos, exposições e cursos, além de uma área dedicada ao Centro de Formação Artística e Tecnológica (Cefart) e à história da instituição. Além disso o site também conta com uma bilheteria online, um diferencial valioso. O design é responsivo, adaptando-se bem a dispositivos móveis, e utiliza uma paleta de cores que facilita a navegação. A usabilidade é aprimorada por menus claros e uma estrutura de navegação intuitiva. A qualidade do conteúdo é alta, com informações detalhadas sobre as atividades e serviços oferecidos.

O Southbank Centre, em Londres, é um dos maiores centros culturais da Europa. Seu site é um exemplo de como integrar diversas funcionalidades de maneira coesa. Oferece uma plataforma robusta para a venda de ingressos, uma agenda detalhada de eventos e espaços dedicados a exposições permanentes e temporárias. O design é moderno, vibrante e visualmente atraente, com uma interface amigável que prioriza a experiência do usuário. A qualidade da informação é rigorosamente mantida, com descrições precisas e atraentes que incentivam a participação do público.

O Centro Cultural São Paulo é um exemplo de como um site pode ser utilizado para engajar a comunidade. Além de informações sobre eventos e exposições, o site inclui um blog com artigos sobre cultura e arte, promovendo um diálogo contínuo com o público. A funcionalidade de bilheteria e inscrição em editais diretamente pelo site facilita o acesso aos serviços oferecidos. O design é limpo e funcional, focando na simplicidade para garantir a fácil navegação. A qualidade da informação é consistente, com atualizações frequentes e conteúdo relevante para os visitantes.

A partir da análise dos exemplos acima, diversas boas práticas e tendências foram identificadas, servindo como diretrizes para o desenvolvimento do site da Casa de Cultura de João Monlevade. As funcionalidades devem ir além da simples apresentação de informações, proporcionando interatividade e facilidades aos usuários. A inclusão de calendários de eventos, áreas dedicadas a artistas locais e blogs são exemplos de como engajar o público e facilitar o acesso aos serviços culturais.

Outra consideração importante é que o design do site deve ser responsivo e adaptável a diversos dispositivos, garantindo uma boa experiência tanto em desktops quanto em dispositivos móveis \cite{almeida2017role}. A usabilidade deve ser uma prioridade, com menus intuitivos, navegação fácil e informações acessíveis. A adoção de padrões de design que facilitem a leitura e a interação é essencial para manter os usuários engajados. A qualidade da informação deve ser mantida com conteúdos detalhados, atualizados e relevantes. Descrições precisas de eventos, perfis de artistas, artigos sobre cultura e arte, e notícias sobre a instituição ajudam a criar um site rico e informativo. Além disso, a linguagem utilizada deve ser clara e acessível, atingindo um público diversificado.
