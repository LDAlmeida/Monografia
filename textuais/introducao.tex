% ----------------------------------------------------------
% Introdução
% ----------------------------------------------------------
\chapter{Introdução}
\label{cap:introducao}
% ----------------------------------------------------------

No mundo contemporâneo, marcado pela conectividade e pelos avanços tecnológicos, a cultura e as artes encontram na tecnologia uma aliada poderosa para sua promoção e difusão. A Casa de Cultura de Monlevade, desempenhando o importante papel de Secretaria Municipal de Cultura na cidade de João Monlevade, tem buscado formas inovadoras de fomentar o talento artístico local. Nesse contexto, o desenvolvimento de uma plataforma de artistas monlevadenses surge como uma iniciativa que visa valorizar e promover a riqueza cultural da região.

No entanto, a criação de uma plataforma cultural esbarra em um desafio recorrente enfrentado por muitas iniciativas culturais no Brasil: a escassez de recursos financeiros. Em um cenário de cortes orçamentários e falta de investimentos, torna-se ainda mais desafiador obter os recursos necessários para viabilizar projetos como esse. É nesse contexto que se busca por alternativas criativas e parcerias gratuitas para tornar possível a concretização da plataforma de artistas monlevadenses.

% ----------------------------
\section{O problema de pesquisa}
\label{sec:problema}

O desenvolvimento de uma plataforma de artistas monlevadenses para a Casa de Cultura esbarra na dificuldade em obter recursos financeiros para sua realização. Em um contexto de restrições orçamentárias e escassez de investimentos na cultura, torna-se desafiador viabilizar a criação de uma plataforma digital que promova a cultura local e valorize os talentos artísticos de Monlevade. A falta de recursos financeiros adequados e sustentáveis se apresenta como um obstáculo para o desenvolvimento e a manutenção dessa iniciativa cultural, comprometendo seu potencial de impacto e alcance na comunidade.

Compreender os desafios relacionados à captação de recursos e buscar alternativas viáveis para financiar o projeto é fundamental para tornar possível a implementação e a continuidade da plataforma de artistas monlevadenses. É necessário encontrar estratégias e principalmente parcerias que permitam superar as restrições orçamentárias e garantir a sustentabilidade financeira dessa plataforma, a fim de fortalecer a cultura local, promover a divulgação das obras de arte e proporcionar uma experiência enriquecedora para o público.

Outro problema a ser vencido é a falta de conhecimento tecnológico por parte dos artistas, criando-se assim a necessidade de garantir a acessibilidade e a usabilidade do site para todos os públicos, sejam esses artistas ou cidadãos comuns. A construção do site deve levar em consideração essa questão e implementar soluções para que as informações sejam passadas de forma integra e de fácil entendimento para todos. A simplicidade e interação de fácil dedução é chave para alcançar esse objetivo.

Além disso, um grande problema que é recorrente quando se trata de sistemas em geral é a manutenibilidade do código. O site e o sistema de gestão deve ser desenvolvido não apenas pensando na interação site-usuário, mas também ser bem estruturado na parte do backend para que não seja abandonado pela sua complexidade. A escalabilidade de um sistema é de grande importancia para sua longevidade, tornando as boas práticas de programação um ponto chave para o sucesso do projeto como um todo.

A facilidade atualização do conteúdo também é outro fator importante a ser considerado. A implementação de um sistema CMS resolve o problema fornecendo uma interface amigável para a publicação de artigos, edição de páginas e controle de conteúdos do todo o site, além de restrigir o acesso a parte administrativa de forma efetiva com a criação de usuários e definição de papéis. A escolha do CoderedCMS, um \textit{fork} do Wagtail, sistema de gestão de conteúdo baseado em Python, atende perfeitamente as demandas da Casa de Cultura e permite que o sistema de gestão para a Escola de Artes, controle de Editais e Vitrine de Artistas esteja no mesmo \textit{codebase}.

Assim, o problema de pesquisa deste trabalho consiste em como superar a dificuldade em obter recursos financeiros para o desenvolvimento da plataforma de artistas monlevadenses, criando essa parceria com a Casa de Cultura que permite a viabilização e sustentabilidade do projeto, garantindo o acesso à cultura e valorizando os talentos artísticos presentes em Monlevade.

\section{Objetivos}
\label{sec:objetivos}

O objetivo geral desse trabalho é justamente contribuir para a solução desse problema, desenvolvendo de uma plataforma digital que funcione como uma vitrine para os talentos artísticos de Monlevade. Através dessa plataforma, será possível não apenas expor e divulgar as obras de arte, mas também proporcionar ao público uma forma acessível e interativa de conhecer e apreciar a produção cultural local, através da publicação de artigos no estilo blog. Além disso, também será desenvolvimento um sistema de gestão para a Escola de Artes, informatizando o fluxo de trabalho. Esse sistema de gestão também auxiliará na publicação de Editais para a casa de cultura, além de gerenciar os outros módulos.

Os objetivos específicos são:

\begin{itemize}
	\item Confecção de um site para a publicação de artigos, implementando um sistema CMS.
	\item Desenvolvimento do sistema de gestão para a Escola de Artes.
	\item Desenvolver uma Vitrine de Artistas, acessível através do site da Casa de Cultura.
	\item Criação de um módulo de editais para o sistema de gestão, também possibilitando acesso as informações no site da Casa de Cultura.
\end{itemize}


\section{Metodologia}
\label{sec:metodologia}

Segundo \cite{moresi2003metodologia}, a metodologia de pesquisa pode ser compreendida como os procedimentos empregados na pesquisa, que incluem uma estratégia, os passos práticos e as técnicas específicas utilizadas para a realização da investigação, ou seja, abrange a coleta de dados, a escolha da amostra, os instrumentos de pesquisa e outras atividades práticas. Para avaliar a eficácia do sistema, foi desenvolvido um estudo de caso com a Casa de Cultura de João Monlevade. Esse processo envolveu a implementação de formulários destinados à avaliação do da usabilidade do sistema. Com a finalidade de garantir a integridade do processo e incentivar a transparência nas respostas, os formulários foram preenchidos de forma anônima, criando assim um ambiente de confiança entre os funcionários e o sistema. Após a coleta de dados, um membro da equipe foi designado para realizar uma análise detalhada dos indicadores de bem-estar obtidos, possibilitando uma avaliação precisa do impacto do sistema.

O objeto de pesquisa deste trabalho é o desenvolvimento de uma plataforma digital dedicada aos artistas monlevadenses, com o intuito de promover a cultura local e valorizar os talentos artísticos presentes na região. A plataforma funcionará como uma vitrine virtual, oferecendo um espaço de exposição e divulgação das obras de arte produzidas pelos artistas locais, abrangendo diversas formas de expressão, como pintura, escultura, fotografia, música, literatura, entre outras.

Os passos para execução deste trabalho são assim definidos:

\begin{itemize}
	\item Fazer uma revisão da literatura sobre os tópicos: Linguagem de Modelagem Unificada (\ac{UML}), tecnologias empregadas no desenvolvimento do trabalho e pesquisas correlatas ao tema deste estudo;
	\item Desenvolvimento do sistema de gestão para a Casa de Cultura, com os módulos de vitrine de artistas, editais e escola de artes;
	\item Desenvolvimento do sistema CMS para publicação de artigos, integrado com o sistema de gestão;
	\item Executar testes no sistema para assegurar que ele funcione corretamente e coletar feedbacks que confirmem sua eficácia.
\end{itemize}

\section{Organização do trabalho}

Neste capítulo, foi definido o contexto do tema tratado neste trabalho, delineando o problema de pesquisa e os objetivos estabelecidos. No Capítulo 2, é feita uma revisão bibliográfica relevante ao tema. O Capítulo 3 consiste no desenvolvimento do trabalho, detalhando a coleta dos requisitos do sistema e alguns diagramas Unified Modeling Language (\ac{UML}). Além disso, este capítulo descreve as interfaces do sistema e o funcionamento de cada tela. No Capítulo 4, é abordada a metodologia utilizada para avaliar o sistema, bem como para mensurar a qualidade do sistema. Os resultados dessa avaliação também são discutidos nesta seção. Por fim, o Capítulo 5 apresenta as conclusões do projeto.
