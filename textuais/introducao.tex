% ----------------------------------------------------------
% Introdução
% ----------------------------------------------------------
\chapter{Introdução}
\label{cap:introducao}
% ----------------------------------------------------------

No mundo contemporâneo, marcado pela conectividade e pelos avanços tecnológicos, a cultura e as artes encontram na tecnologia uma aliada poderosa para sua promoção e difusão \cite{gasparetto2014arte}. A Casa de Cultura de Monlevade, desempenhando o importante papel de Secretaria Municipal de Cultura na cidade de João Monlevade, tem buscado formas inovadoras de fomentar o talento artístico local. Assim, o desenvolvimento de uma plataforma de artistas monlevadenses surge como uma iniciativa que visa valorizar e promover a riqueza cultural da região.

No entanto, a criação de uma plataforma cultural esbarra em um desafio recorrente enfrentado por muitas iniciativas culturais no Brasil: a escassez de recursos financeiros. De acordo com \citeonline{bernardi2021financiamento}, as políticas de financiamento da cultura no Brasil enfrentam sérias limitações, resultando em dificuldades para garantir a sustentabilidade financeira de muitas iniciativas culturais, que dependem de um orçamento público restrito e modelos de financiamento insuficientes. Em um cenário de cortes orçamentários e falta de investimentos, torna-se ainda mais desafiador obter os recursos necessários para viabilizar projetos como esse. É nesse contexto que se busca por alternativas criativas e parcerias gratuitas para tornar possível a concretização da plataforma de artistas monlevadenses.

Além de servir como um meio de divulgação, a plataforma se propõe a ser uma ponte entre os artistas locais e o público, facilitando o acesso à arte produzida na região e permitindo que os artistas ganhem maior visibilidade. Em uma cidade como João Monlevade, cuja cena cultural é rica, mas pouco explorada fora de seus limites, uma ferramenta como essa tem o potencial de fortalecer o setor artístico e criar novas oportunidades de valorização e reconhecimento. Ao reunir informações sobre os artistas, seus portfólios e produções, a plataforma oferece um espaço centralizado para que o público possa conhecer e se conectar com o que é produzido na cidade.

No entanto, a implementação de uma plataforma dessa magnitude enfrenta desafios que vão além da simples escassez de recursos. Há também a necessidade de garantir que ela seja funcional, acessível e atrativa tanto para os artistas quanto para o público. A usabilidade e a intuitividade do sistema são aspectos cruciais para que o projeto seja bem-sucedido, considerando que muitos dos artistas locais podem não ter familiaridade com tecnologias digitais complexas. Portanto, o desenvolvimento de uma interface amigável, que promova a participação ativa dos usuários, é essencial para o engajamento.

Paralelamente aos desafios, a criação dessa plataforma também traz consigo diversas oportunidades. A iniciativa pode fortalecer a identidade cultural da cidade, promovendo uma rede de colaboração entre artistas, instituições culturais e o público. Além disso, a plataforma tem o potencial de fomentar o turismo cultural na cidade, atraindo visitantes interessados em conhecer o trabalho dos artistas monlevadenses e participar dos eventos promovidos pela Casa de Cultura.

Dessa forma, o desenvolvimento da plataforma se apresenta não apenas como uma solução tecnológica, mas também como uma estratégia para a valorização da cultura local, possibilitando que os artistas tenham maior autonomia na divulgação de suas obras e que o público tenha acesso facilitado à produção cultural de João Monlevade.

% ----------------------------
\section{O problema de pesquisa}
\label{sec:problema}

O desenvolvimento de uma plataforma de artistas monlevadenses para a Casa de Cultura esbarra na dificuldade em obter recursos financeiros para sua realização. Em um contexto de restrições orçamentárias e escassez de investimentos na cultura, torna-se desafiador viabilizar a criação de uma plataforma digital que promova a cultura local e valorize os talentos artísticos de Monlevade. A falta de recursos financeiros adequados e sustentáveis se apresenta como um obstáculo para o desenvolvimento e a manutenção dessa iniciativa cultural, comprometendo seu potencial de impacto e alcance na comunidade.

Compreender os desafios relacionados à captação de recursos e buscar alternativas viáveis para financiar o projeto é fundamental para tornar possível a implementação e a continuidade da plataforma de artistas monlevadenses. É necessário encontrar estratégias e principalmente parcerias que permitam superar as restrições orçamentárias e garantir a sustentabilidade financeira dessa plataforma, a fim de fortalecer a cultura local, promover a divulgação das obras de arte e proporcionar uma experiência enriquecedora para o público.

Um desafio que pode surgir é a diversidade de familiaridade com tecnologia entre os artistas, o que torna importante priorizar a acessibilidade e a usabilidade do site para todos os públicos, sejam esses artistas, ou cidadãos comuns. A construção do site deve levar em consideração essa questão e implementar soluções para que as informações sejam passadas de forma integra e de fácil entendimento para todos. A simplicidade e interação de fácil dedução é chave para alcançar esse objetivo.

Um fator que não pode ser subestimado é o engajamento da própria comunidade artística e do público local. Mesmo com uma plataforma bem estruturada e acessível, a adesão ao projeto depende da vontade dos artistas de participar e da disposição do público de interagir com o conteúdo disponível. Isso inclui não apenas a divulgação da plataforma, mas também a criação de estratégias que incentivem os artistas a participarem ativamente, mostrando-lhes os benefícios e oportunidades que a plataforma pode oferecer. Sem um forte engajamento da comunidade, o impacto da plataforma pode ser limitado.

Além disso, um grande problema recorrente em sistemas em geral é a manutenibilidade do código. O site e o sistema de gestão devem ser desenvolvidos não apenas pensando na interação site-usuário, mas também com uma estrutura sólida no \textit{backend}, que é a parte do sistema responsável por gerenciar o funcionamento interno, como o armazenamento e processamento de dados, além da lógica de negócio. Um \textit{backend} bem estruturado garante que o projeto não seja abandonado devido à sua complexidade. A escalabilidade do sistema, ou seja, a capacidade de expandir e adaptar o código para novas funcionalidades e maior volume de dados, é essencial para sua longevidade. Dessa forma, a adoção de boas práticas de programação se torna um fator crucial para o sucesso do projeto como um todo.

A facilidade de atualização do conteúdo também é outro fator importante a ser considerado. A implementação de um sistema \ac{CMS} resolve o problema fornecendo uma interface amigável para a publicação de artigos, edição de páginas e controle de conteúdos de todo o site, além de restringir o acesso a parte administrativa de forma efetiva com a criação de usuários e definição de papéis \cite{Baker2013}. 

A escolha do Wagtail CRX \cite{WagtailCRX}, que é um \textit{fork} do Wagtail \cite{Wagtail} — ou seja, uma versão derivada do Wagtail original, onde foi criado um novo projeto com base no código existente —, atende perfeitamente às demandas da Casa de Cultura. Essa escolha permite que o sistema de gestão para a Escola de Artes (projeto da Casa de Cultura de João Monlevade), o controle de editais para artistas locais e a Vitrine de Artistas estejam no mesmo \textit{codebase}, ou seja, compartilhando a mesma base de código. Isso facilita a manutenção, a integração e o desenvolvimento contínuo de todas essas funcionalidades dentro de uma única plataforma.

Considerando a necessidade de manter o conteúdo da plataforma atualizado e relevante, surge o desafio de implementar processos contínuos de curadoria e gerenciamento de conteúdo. A manutenção da qualidade e pertinência das informações requer um esforço constante de coleta e atualização de dados sobre os artistas, seus trabalhos e eventos culturais. Isso demanda tempo, equipe qualificada e um sistema eficiente para facilitar a atualização, o que pode ser dificultado pela falta de recursos humanos e tecnológicos. Além disso, sem um planejamento de longo prazo, a plataforma corre o risco de se tornar obsoleta ou desatualizada, perdendo seu valor como ferramenta de promoção cultural.

Assim, o problema de pesquisa deste trabalho consiste em como desenvolver uma plataforma que valorize os artistas monlevadenses, garantindo o acesso à cultura e promovendo os talentos artísticos presentes em João Monlevade.

\section{Objetivos}
\label{sec:objetivos}

O objetivo geral deste trabalho é desenvolver uma plataforma digital que funcione como uma vitrine para os talentos artísticos de Monlevade, contribuindo para a promoção e valorização da cultura local. Através dessa plataforma, os artistas terão um espaço dedicado para expor e divulgar suas obras, e o público poderá acessar de maneira interativa e acessível o vasto acervo cultural da cidade. Além disso, será implementado um sistema de gestão informatizado para a Escola de Artes, que facilitará o controle de atividades internas e a publicação de editais, promovendo uma maior organização e eficiência no funcionamento da Casa de Cultura.

Os objetivos específicos são:

\begin{itemize}
	\item Desenvolvimento de um site com sistema \ac{CMS} para publicação de artigos: Implementar um \ac{CMS} que permitirá a fácil publicação e atualização de artigos e notícias relacionadas à cultura local. O objetivo é criar um ambiente digital dinâmico e constantemente atualizado, que promova o engajamento do público.

	\item Desenvolvimento de um sistema de gestão para a Escola de Artes: Criar um sistema integrado que auxilie na administração das atividades da Escola de Artes, como o gerenciamento de alunos, turmas, cursos e eventos. Esse sistema deve ser acessível e fácil de usar para os gestores, com o objetivo de otimizar o fluxo de trabalho e melhorar a eficiência operacional da escola.

	\item Criação da Vitrine de Artistas: Desenvolver uma funcionalidade dentro do site da Casa de Cultura que permita aos artistas locais cadastrar suas obras, biografias e informações relevantes, criando um catálogo interativo acessível ao público. O objetivo é garantir uma plataforma inclusiva, onde todos os artistas possam expor seus trabalhos de forma organizada e fácil de navegar, promovendo maior visibilidade para a produção artística de Monlevade.

	\item Implementação do módulo de Editais: Desenvolver um módulo que facilite a criação, gerenciamento e publicação de editais culturais. Esse sistema permitirá a Casa de Cultura gerir os processos seletivos de maneira mais ágil e transparente, com acesso público às informações por meio do site. O objetivo é assegurar que a divulgação de oportunidades culturais seja feita de forma clara e acessível.

	\item Avaliação das funcionalidades implementadas: As funcionalidades desenvolvidas serão avaliadas com base em critérios como usabilidade, acessibilidade e eficiência. Para garantir a eficácia do sistema, serão realizados testes automatizados que analisarão o desempenho e a conformidade do site. O objetivo é alcançar uma plataforma funcional e eficaz, que otimize o gerenciamento cultural e fortaleça o vínculo entre a Casa de Cultura e a comunidade artística.
\end{itemize}


\section{Metodologia}
\label{sec:metodologia}

Segundo \citeonline{moresi2003metodologia}, a metodologia de pesquisa pode ser compreendida como os procedimentos empregados na pesquisa, que incluem uma estratégia, os passos práticos e as técnicas específicas utilizadas para a realização da investigação, ou seja, abrange a coleta de dados, a escolha da amostra, os instrumentos de pesquisa e outras atividades práticas. Neste trabalho, a análise será feita com base no processo de desenvolvimento e implementação da plataforma, avaliando sua adequação em termos de acessibilidade, usabilidade e impacto positivo para os artistas monlevadenses. 

Para isso, serão utilizadas ferramentas \textit{online} que auxiliam na avaliação do desempenho e da conformidade do sistema com os padrões da web, bem como na segurança. Page Speed Insights analisa a velocidade de carregamento e sugere melhorias de usabilidade e acessibilidade, enquanto o W3C Validator verifica a conformidade do código com os padrões da web. GTMetrix oferece relatórios detalhados sobre o desempenho do site, e Pentest Tools e ImmuniWeb são utilizados para testar a segurança da plataforma. Essas ferramentas visam garantir que o sistema atenda às necessidades do público-alvo e promova a valorização dos artistas locais.

Os passos para execução deste trabalho são assim definidos:

\begin{itemize}
	\item Fazer uma revisão da literatura sobre os tópicos: \ac{UML} como definido por \citeonline{booch1999uml}, tecnologias empregadas no desenvolvimento do trabalho e pesquisas correlatas ao tema deste estudo;
	\item Desenvolvimento do sistema de gestão para a Casa de Cultura, com os módulos de vitrine de artistas, editais e escola de artes;
	\item Desenvolvimento do sistema \ac{CMS} para publicação de artigos, integrado com o sistema de gestão;
	\item Executar testes no sistema para assegurar que ele funcione corretamente e confirmar sua eficácia.
\end{itemize}

\section{Organização do trabalho}

Neste capítulo, foi definido o contexto do tema tratado neste trabalho, delineando o problema de pesquisa e os objetivos estabelecidos. No Capítulo 2, é feita uma revisão bibliográfica relevante ao tema. O Capítulo 3 consiste no desenvolvimento do trabalho, detalhando a coleta dos requisitos do sistema e alguns diagramas \ac{UML}. Além disso, este capítulo descreve as interfaces do sistema e o funcionamento de cada tela. No Capítulo 4, é abordada a metodologia utilizada para avaliar o sistema, bem como para mensurar a qualidade do sistema. Os resultados dessa avaliação também são discutidos nesta seção. Por fim, o Capítulo 5 apresenta as conclusões do projeto.
