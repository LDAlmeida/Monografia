% ----------------------------------------------------------
% Introdução
% ----------------------------------------------------------
\chapter{Introdução}
\label{cap:introducao}
% ----------------------------------------------------------

No mundo contemporâneo, marcado pela conectividade e pelos avanços tecnológicos, a cultura e as artes encontram na tecnologia uma aliada poderosa para sua promoção e difusão. A Casa de Cultura de Monlevade, desempenhando o importante papel de Secretaria Municipal de Cultura na cidade de João Monlevade, tem buscado formas inovadoras de fomentar o talento artístico local. Nesse contexto, o desenvolvimento de uma plataforma de artistas monlevadenses surge como uma iniciativa que visa valorizar e promover a riqueza cultural da região.

No entanto, a criação de uma plataforma cultural esbarra em um desafio recorrente enfrentado por muitas iniciativas culturais no Brasil: a escassez de recursos financeiros. Em um cenário de cortes orçamentários e falta de investimentos, torna-se ainda mais desafiador obter os recursos necessários para viabilizar projetos como esse. É nesse contexto que se busca por alternativas criativas e parcerias gratuitas para tornar possível a concretização da plataforma de artistas monlevadenses.

% ----------------------------

\section{Elaboração do capítulo}

Este capítulo apresenta o seu trabalho. Você deve contextualizar o problema abordado, descrever os objetivos gerais e específicos, apresentar a metodologia e como o trabalho está estruturado.

\section{O problema de pesquisa}
\label{sec:problema}

O desenvolvimento de uma plataforma de artistas monlevadenses para a Casa de Cultura esbarra na dificuldade em obter recursos financeiros para sua realização. Em um contexto de restrições orçamentárias e escassez de investimentos na cultura, torna-se desafiador viabilizar a criação de uma plataforma digital que promova a cultura local e valorize os talentos artísticos de Monlevade. A falta de recursos financeiros adequados e sustentáveis se apresenta como um obstáculo para o desenvolvimento e a manutenção dessa iniciativa cultural, comprometendo seu potencial de impacto e alcance na comunidade.

Compreender os desafios relacionados à captação de recursos e buscar alternativas viáveis para financiar o projeto é fundamental para tornar possível a implementação e a continuidade da plataforma de artistas monlevadenses. É necessário encontrar estratégias e parcerias que permitam superar as restrições orçamentárias e garantir a sustentabilidade financeira dessa plataforma, a fim de fortalecer a cultura local, promover a divulgação das obras de arte e proporcionar uma experiência enriquecedora para o público.

Assim, o problema de pesquisa deste trabalho consiste em como superar a dificuldade em obter recursos financeiros para o desenvolvimento da plataforma de artistas monlevadenses, buscando estratégias de captação de recursos e parcerias que permitam sua viabilização e sustentabilidade financeira, garantindo o acesso à cultura e valorizando os talentos artísticos presentes em Monlevade.

\section{Objetivos}
\label{sec:objetivos}


O objetivo principal desse trabalho é justamente contribuir para a solução desse problema, fornecendo subsídios teóricos e práticos para o desenvolvimento de uma plataforma digital que funcione como uma vitrine para os talentos artísticos de Monlevade. Através dessa plataforma, será possível não apenas expor e divulgar as obras de arte, mas também proporcionar ao público uma forma acessível e interativa de conhecer e apreciar a produção cultural local.

Neste trabalho, serão explorados conceitos de gestão cultural, empreendedorismo e tecnologia, buscando embasamento teórico para a compreensão dos desafios e possíveis soluções relacionadas ao desenvolvimento da plataforma de artistas monlevadenses. Além disso, serão analisadas estratégias de captação de recursos e parcerias que possam viabilizar a implementação e sustentabilidade financeira do projeto.

Ao final deste estudo, espera-se que o resultado obtido possa servir como um valioso instrumento para a Casa de Cultura de Monlevade, fornecendo subsídios práticos para a criação e disponibilização da plataforma, promovendo assim a valorização da cultura local e estimulando a criação artística na região.

\section{Metodologia}
\label{sec:metodologia}

O objeto de pesquisa deste trabalho é o desenvolvimento de uma plataforma digital dedicada aos artistas monlevadenses, com o intuito de promover a cultura local e valorizar os talentos artísticos presentes na região. A plataforma funcionará como uma vitrine virtual, oferecendo um espaço de exposição e divulgação das obras de arte produzidas pelos artistas locais, abrangendo diversas formas de expressão, como pintura, escultura, fotografia, música, literatura, entre outras

Os passos para execução deste trabalho são assim definidos:

\begin{itemize}
	\item Revisão da literatura
	\item Desenvolvimento do sistema
	\item Validação do sistema
	\item Análise e discussão do sistema
\end{itemize}

\section{Organização do trabalho}

\textbf{É importante observar que a estrutura é apresentada a partir do próximo capítulo. O capítulo de Introdução não deve compor esta descrição. Além disso, sempre que você fizer referência à algum item específico, a inicial deve ser maiúscula. Por exemplo, Capítulo 2, Tabela 5, Figura 1, dentre outros.}

O restante deste trabalho é organizado como se segue. O Capítulo~\ref{cap:revisao} apresenta...
