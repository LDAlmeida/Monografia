% ----------------------------------------------------------
% Conclusão
% ----------------------------------------------------------
\chapter[Conclusão]{Conclusão}
%\addcontentsline{toc}{chapter}{Conclusão}
% ---

O desenvolvimento da plataforma de artistas monlevadenses para a Casa de Cultura de João Monlevade resultou em uma solução robusta e inovadora que atende às necessidades da instituição e dos artistas locais. A proposta inicial foi criar um sistema que reunisse, em um só ambiente, a divulgação de notícias culturais, a gestão de editais, o acompanhamento das atividades da Escola de Artes e, principalmente, a criação de uma vitrine digital para os artistas da cidade poderem expor seus trabalhos. Esses objetivos foram plenamente atingidos e validados durante os testes, o que demonstrou a funcionalidade e eficácia da plataforma para essas finalidades.

A relevância do site para a Casa de Cultura de João Monlevade se evidencia em vários aspectos. A vitrine digital dos artistas, por exemplo, representa um avanço significativo para a promoção da cultura local. Artistas de diferentes áreas — como música, teatro, artes plásticas e dança — agora possuem um espaço online dedicado à exposição de seus trabalhos. Esse ambiente proporciona uma maior visibilidade para os artistas da cidade, especialmente aqueles que ainda não possuem um portfólio digital consolidado, democratizando o acesso à divulgação de arte e cultura. Além disso, a inclusão de funcionalidades que permitem a inserção de links, vídeos e imagens ampliou as possibilidades de exposição, tornando a plataforma mais interativa e visualmente atrativa tanto para artistas quanto para o público.

Apesar dos resultados positivos, o desenvolvimento da plataforma abre portas para futuras pesquisas e melhorias. Entre as sugestões para trabalhos futuros, está a possibilidade de integrar novas tecnologias que possam aprimorar a experiência do usuário. Um exemplo disso seria a utilização de algoritmos de recomendação baseados em aprendizado de máquina, que sugeririam artistas ou eventos culturais de acordo com os interesses do usuário. Tal funcionalidade poderia aumentar o engajamento do público com a plataforma, personalizando a experiência de navegação e promovendo uma maior interação entre os artistas e o público.

Outra área que merece destaque para pesquisas futuras é a ampliação da acessibilidade da plataforma. Embora o projeto atual já siga boas práticas de usabilidade, como a organização clara de informações e o design responsivo, há espaço para explorar soluções mais inclusivas. Isso inclui a implementação de tecnologias assistivas que garantam o acesso de pessoas com deficiências, como leitores de tela para pessoas com deficiência visual e comandos de voz para navegação. Dessa forma, a plataforma poderia atender a um público ainda mais amplo, cumprindo seu papel de ser um espaço inclusivo e democrático para a promoção da arte e cultura.

Em termos de expansão de funcionalidades, uma possibilidade interessante seria o desenvolvimento de uma seção de monetização, onde artistas poderiam vender suas obras ou serviços diretamente pela plataforma. Isso poderia ser implementado na forma de um marketplace, permitindo que os artistas locais comercializem seus trabalhos de forma mais acessível e, ao mesmo tempo, gerando uma nova fonte de renda para a Casa de Cultura. Outro aspecto a ser explorado seria o estabelecimento de parcerias com outras instituições culturais, criando eventos colaborativos ou exposições itinerantes que utilizassem a plataforma como um meio de divulgação e gestão.

Por fim, a pesquisa futura poderia também focar no impacto socioeconômico da plataforma. A vitrine digital dos artistas tem o potencial de se transformar em uma ferramenta relevante para o turismo cultural em João Monlevade, atraindo visitantes e apreciadores de arte interessados em conhecer o trabalho dos artistas locais. Estudar os efeitos dessa plataforma no desenvolvimento cultural e econômico da cidade seria um campo de pesquisa interessante, além de fornecer dados importantes para futuras políticas culturais na região.

Dessa forma, o projeto não apenas cumpre seu papel de modernizar as operações da Casa de Cultura, como também oferece uma base sólida para inovações que podem impactar a comunidade artística e cultural de João Monlevade de maneira significativa e duradoura.
