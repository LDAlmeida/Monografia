\begin{resumo}[Abstract]
 \begin{otherlanguage*}{english}
  In an increasingly connected world, technology is an invaluable ally in promoting culture and the arts. In this sense, the Casa de Cultura de Monlevade has been seeking new ways to foster local artistic talent, and the development of a platform for artists from Monlevade is an initiative that has stood out in this scenario. The problem that this work aims to contribute to solving is a demand pointed out by the current Coordinator of the Casa de Cultura, who sought partnerships at UFOP to develop solutions. Given this context, it is clear that the development of the platform for artists from Monlevade is an important step towards strengthening the local cultural scene. The system was subjected to a case study with the Casa de Cultura, at which time the responses were collected and analyzed using a scale of 1 to 5, calculating the average to determine the level of quality of the system. The results revealed that expectations were met, highlighting the usability of the system, the improvement of the employees' workflow, and the democratization of access to information.

   \vspace{\onelineskip}

   \noindent
   \textbf{Key-words}: 
   Artists' platform.
   Development of a cultural platform.
   Cultural House.
   Monlevade artists.
   Cultural management.
   WEB system
   Technology in culture.
   Management software
   Promotion of local artists.
 \end{otherlanguage*}
\end{resumo}