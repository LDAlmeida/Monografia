\begin{resumo}[Abstract]
 \begin{otherlanguage*}{english}

  In an increasingly connected world, technology has become an invaluable ally in promoting culture and the arts. In this sense, the Casa de Cultura de João Monlevade has been seeking new ways to foster local artistic talent, and the development of a platform for Monlevade artists is an initiative that has stood out in this scenario. The problem that this work aims to assist is a demand pointed out by the current Coordinator of the Casa de Cultura, who sought partnerships with UFOP to develop their own website. Given this context, we believe that the development of the platform for Monlevade artists is an important step towards strengthening the local cultural scene. Although the tests carried out were conducted in a simulated environment and not directly implemented by the Casa de Cultura, the results were promising. The evaluations demonstrated that the platform will be able to meet the institution's expectations, offering efficient usability, improving the workflow of employees, and democratizing access to information about the Casa de Cultura and local artists.

   \vspace{\onelineskip}

   \noindent
   \textbf{Key-words}: 
   Artists' platform.
   Development of a cultural platform.
   Cultural House.
   Monlevade artists.
   Cultural management.
   WEB system
   Technology in culture.
   Management software
   Promotion of local artists.
 \end{otherlanguage*}
\end{resumo}