% resumo em português
\setlength{\absparsep}{18pt} % ajusta o espaçamento dos parágrafos do resumo

\begin{resumo}
Em um mundo cada vez mais conectado, a tecnologia se apresenta como uma aliada inestimável na promoção da cultura e das artes. Nesse sentido, a Casa de Cultura de Monlevade tem buscado novas formas de fomentar o talento artístico local, e o desenvolvimento de uma plataforma de artistas monlevadenses é uma iniciativa que tem se destacado nesse cenário. Cabe salientar que na cidade de João Monlevade, a Casa de Cultura cumpre um  papel de Secretaria Municipal de Cultura. O problema que este trabalho pretende contribuir na solução é uma demanda apontada pela atual Coordenadora da Casa de Cultura que buscou parcerias na Ufop para o desenvolvimento de soluções. Diante desse contexto, é evidente que o desenvolvimento da plataforma de artistas monlevadenses é um passo importante para o fortalecimento da cena cultural local. Para Raman & Bansal (2014, p. 15): "Um portal para arte e artistas pode desempenhar um papel importante na promoção da cultura local e na criação de um espaço para a exibição e venda de obras de arte, além de fornecer informações sobre eventos culturais na região.”,  e é com esse objetivo em mente que a Casa de Cultura tem trabalhado, buscando parcerias e recursos para tornar essa ideia realidade.


 \textbf{Palavras-chaves}:     Plataforma de artistas.
    Desenvolvimento de plataforma cultural.
    Casa de Cultura.
    Artistas monlevadenses.
    Gestão cultural.
    Economia criativa.
    Tecnologia na cultura.
    Marketing cultural.
    Empreendedorismo cultural.
    Promoção de artistas locais.
\end{resumo}
