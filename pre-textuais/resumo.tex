% resumo em português
\setlength{\absparsep}{18pt} % ajusta o espaçamento dos parágrafos do resumo

\begin{resumo}
Em um mundo cada vez mais conectado, a tecnologia se apresenta como uma aliada inestimável na promoção da cultura e das artes. Nesse sentido, a Casa de Cultura de Monlevade tem buscado novas formas de fomentar o talento artístico local, e o desenvolvimento de uma plataforma de artistas monlevadenses é uma iniciativa que tem se destacado nesse cenário. O problema que este trabalho pretende contribuir na solução é uma demanda apontada pela atual Coordenadora da Casa de Cultura que buscou parcerias na UFOP para o desenvolvimento de soluções. Diante desse contexto, é evidente que o desenvolvimento da plataforma de artistas monlevadenses é um passo importante para o fortalecimento da cena cultural local. O sistema foi submetido a um estudo de caso com a Casa de Cultura, na ocasião coletou-se e analisou-se as respostas utilizando uma escala de 1 a 5, calculando a média para determinar o nível de qualidade do sistema. Os resultados revelaram o cumprimento das expectativas, destacando-se a usabilidade do sistema, o aprimoramento do fluxo de trabalho dos funcionários e a democratização do acesso à informação.


 \textbf{Palavras-chaves}:     
    Plataforma de artistas.
    Desenvolvimento de plataforma cultural.
    Casa de Cultura.
    Artistas monlevadenses.
    Gestão cultural.
    Sistema WEB
    Tecnologia na cultura.
    Software de gestão
    Promoção de artistas locais.
\end{resumo}
