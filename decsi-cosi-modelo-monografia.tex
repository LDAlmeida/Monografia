%% -----------------------------------------------------------------------
%%
%% MODELO DE MONOGRAFIA DO DECSI
%%
%% -----------------------------------------------------------------------
%% Versions and updates:
%% = Glauber M. Cabral, https://github.com/glaubersp (v1.0)
%% = Oto Braz Assunção, https://github.com/otobraz (v1.1)
%% 		- Adequações à nova resolução do COSI.
%% = Fernando B. Oliveira, fernando@decea.ufop.br (v1.2)
%% 		- Correções na Folha de Aprovação.
%%		- Mudanças na capa
%% 		- Inclusão do Termo de Responsabilidade e da Ata de Defesa atualizada.
%% 		- Estrutura do documento LaTeX
%% = Fernando B. Oliveira, fboliveira@ufop.edu.br (v1.3)
%%    - Correções conforme as alterações da resolução do COSI de 20/03/2018 (update 21/10/2018).
%%    - Retirada da Ata
%%    -
%% = Contribuições: COSI e Biblioteca JM
%% -----------------------------------------------------------------------
%% ESTE MODELO FOI BASEADO EM:
%% -----------------------------------------------------------------------
%% abtex2-modelo-trabalho-academico.tex, v-1.9.6 laurocesar
%% Copyright 2012-2016 by abnTeX2 group at http://www.abntex.net.br/
%%
%% This work may be distributed and/or modified under the
%% conditions of the LaTeX Project Public License, either version 1.3
%% of this license or (at your option) any later version.
%% The latest version of this license is in
%%   http://www.latex-project.org/lppl.txt
%% and version 1.3 or later is part of all distributions of LaTeX
%% version 2005/12/01 or later.
%%
%% This work has the LPPL maintenance status `maintained'.
%%
%% The Current Maintainer of this work is the abnTeX2 team, led
%% by Lauro César Araujo. Further information are available on
%% http://www.abntex.net.br/
%%
%% This work consists of the files abntex2-modelo-trabalho-academico.tex,
%% abntex2-modelo-include-comandos and abntex2-modelo-references.bib
%%
% ------------------------------------------------------------------------
% ------------------------------------------------------------------------
% abnTeX2: Modelo de Trabalho Academico (tese de doutorado, dissertacao de
% mestrado e trabalhos monograficos em geral) em conformidade com
% ABNT NBR 14724:2011: Informacao e documentacao - Trabalhos academicos -
% Apresentacao
% ------------------------------------------------------------------------
% ------------------------------------------------------------------------

\documentclass[
  % -- opções da classe memoir --
  12pt,       % tamanho da fonte
  openright,      % capítulos começam em pág ímpar (insere página vazia caso preciso)
  oneside,      % para impressão em verso e anverso. Oposto a oneside
  a4paper,      % tamanho do papel.
  % -- opções da classe abntex2 --
  %chapter=TITLE,   % títulos de capítulos convertidos em letras maiúsculas
  %section=TITLE,   % títulos de seções convertidos em letras maiúsculas
  %subsection=TITLE,  % títulos de subseções convertidos em letras maiúsculas
  %subsubsection=TITLE,% títulos de subsubseções convertidos em letras maiúsculas
  % -- opções do pacote babel --
  english,      % idioma adicional para hifenização
  french,        % idioma adicional para hifenização
  spanish,     % idioma adicional para hifenização
  brazil        % o último idioma é o principal do documento
  ]{abntex2-decsi}

% ----------------------------------------------------------
% PACOTES E CONFIGURAÇÕES GERAIS
%
% Insira no arquivos os pacotes utilizados e as respectivas configurações.
%
\input{./config/pacotes.tex}
% ----------------------------------------------------------

% ----------------------------------------------------------
% DADOS PESSOAIS, COMPOSIÇÃO DA BANCA E DEFESA
%
% Insira no arquivos os dados acerca do trabalho,
% da composição da banca e da defesa.
%
% ---
% Informações de dados para CAPA e FOLHA DE ROSTO
% ---
\titulo{Desenvolvimento de uma plataforma de artistas monlevadenses para a Casa de Cultura}
\autor{Lucas Duarte Almeida}
\local{João Monlevade}
\dia{18}
\mes{10}
\ano{2024} % Indicar apenas o ano da monografia

% ---
% Dados da Universidade e Curso
% ---
\instituicao{Universidade Federal de Ouro Preto}
\instituto{Instituto de Ciências Exatas e Aplicadas}
\departamento{Departamento de Computação e Sistemas}
\colegiado{Colegiado de Sistemas de Informação}
\curso{Sistemas de Informação}
\grau{Bacharel em Sistemas de Informação}
\disciplina{CSI999 -- Trabalho de Conclusão de Curso II}

% ---
% Dados de Orientação e Banca de Defesa
% ---
\orientador{Mateus Ferreira Satler}
\orientadorTitulacao{Titulação}
\orientadorDepartamento{DECSI -- UFOP}

% ---
% Informações sobre a data de defesa - ata
% ---
\localdefesa{C304}
\horadefesa{17:00}

% ---
% Tipo de Trabalho
% ---
\tipotrabalho{Monografia (graduação)}

% ---
% Gênero do(a) orientador(a)
% ---
%\renewcommand{\orientadorname}{Orientadora:}
% ---
% Gênero do(a) orientador(a)
% ---
%\renewcommand{\coorientadorname}{Coorientadora:}

% ----------------------------------------------------------

% ----
% Início do documento
% ----
\begin{document}

% Seleciona o idioma do documento (conforme pacotes do babel)
%\selectlanguage{english}
\selectlanguage{brazil}

% Retira espaço extra obsoleto entre as frases.
\frenchspacing

% ----------------------------------------------------------
% ELEMENTOS PRÉ-TEXTUAIS
% ----------------------------------------------------------
% \pretextual

% ---
% Capa
% ---
\imprimircapa%
% ---

% ---
% Folha de rosto
% (o * indica que haverá a ficha bibliográfica)
% ---
\imprimirfolhaderosto*
% ---

% ---
% Inserir a ficha bibliografica
% ---
\include{./pre-textuais/fichaCatalografica}
% ---

% ---
% Inserir errata, caso necessário
% ---
\include{./pre-textuais/errata}
% ---

% ---
% Inserir folha de aprovação - gerada por meio do 
% ---

\include{./pre-textuais/folhaDeAprovacao}

% ---

% ---

%\hrule
\pagebreak

% ---
% Dedicatória
% ---
\begin{dedicatoria}
   \vspace*{\fill}
   \centering
   \noindent
   \textit{Este trabalho é dedicado à todos os artistas de João Monlevade.} \vspace*{\fill}
\end{dedicatoria}
% ---

% ---

% ---
% Agradecimentos
% ---
\begin{agradecimentos}

Agradeço aos meus pais, minha família e meus amigos por me ajudarem na minha trajetória.

\end{agradecimentos}
% ---

% ---
% Epígrafe
%\include{./pre-textuais/epigrafe}
% ---

% ---
% RESUMOS
% ---
% resumo em português
% resumo em português
\setlength{\absparsep}{18pt} % ajusta o espaçamento dos parágrafos do resumo

\begin{resumo}
Em um mundo cada vez mais conectado, a tecnologia se apresenta como uma aliada inestimável na promoção da cultura e das artes. Nesse sentido, a Casa de Cultura de João Monlevade tem buscado novas formas de fomentar o talento artístico local, e o desenvolvimento de uma plataforma de artistas monlevadenses é uma iniciativa que tem se destacado nesse cenário. O problema que este trabalho pretende contribuir na solução é uma demanda apontada pela atual coordenadora da Casa de Cultura que buscou parcerias na UFOP para o desenvolvimento de um \textit{website} próprio. Diante desse contexto, acredita-se que o desenvolvimento da plataforma de artistas monlevadenses é um passo importante para o fortalecimento da cena cultural local. Embora os testes realizados tenham sido conduzidos em um ambiente simulado, e não diretamente implementados pela Casa de Cultura, os resultados foram promissores. As avaliações demonstraram que a plataforma poderá atender as expectativas da instituição, oferecendo usabilidade eficiente, melhoria no fluxo de trabalho dos funcionários e democratização do acesso à informação sobre a Casa de Cultura e dos artistas locais.


 \textbf{Palavras-chaves}:     
    Plataforma de artistas.
    Desenvolvimento de plataforma cultural.
    Casa de Cultura.
    Artistas monlevadenses.
    Gestão cultural.
    Sistema WEB
    Tecnologia na cultura.
    Software de gestão
    Promoção de artistas locais.
\end{resumo}


% resumo em inglês
\begin{resumo}[Abstract]
 \begin{otherlanguage*}{english}

  In an increasingly connected world, technology has become an invaluable ally in promoting culture and the arts. In this sense, the Casa de Cultura de João Monlevade has been seeking new ways to foster local artistic talent, and the development of a platform for Monlevade artists is an initiative that has stood out in this scenario. The problem that this work aims to assist is a demand pointed out by the current coordinator of the Casa de Cultura, who sought partnerships with UFOP to develop their own website. Given this context, we believe that the development of the platform for Monlevade artists is an important step towards strengthening the local cultural scene. Although the tests carried out were conducted in a simulated environment and not directly implemented by the Casa de Cultura, the results were promising. The evaluations demonstrated that the platform will be able to meet the institution's expectations, offering efficient usability, improving the workflow of employees, and democratizing access to information about the Casa de Cultura and local artists.

   \vspace{\onelineskip}

   \noindent
   \textbf{Key-words}: 
   Artists' platform.
   Development of a cultural platform.
   Cultural House.
   Monlevade artists.
   Cultural management.
   WEB system
   Technology in culture.
   Management software
   Promotion of local artists.
 \end{otherlanguage*}
\end{resumo}
% ---

% ---
% inserir lista de ilustrações
% ---
\pdfbookmark[0]{\listfigurename}{lof}
\listoffigures*
\cleardoublepage%
% ---

% ---
% inserir lista de tabelas
%\pdfbookmark[0]{\listtablename}{lot}
%\listoftables*
%\cleardoublepage%
% ---

% ---
% inserir lista de abreviaturas e siglas
% ---
% Este é o modelo do abnTeX2
% ---
% inserir lista de abreviaturas e siglas
% ---
%\begin{siglas}
%  \item[ABNT] Associação Brasileira de Normas Técnicas
%  \item[abnTeX] ABsurdas Normas para TeX
%\end{siglas}
% ---

%
% Outra opção é utilizar o pacote Acronym (mais funcional).
%

%\ac{SW}  	SW documents are beautifully typeset.
%\acf{SW} 	Scientific Word (SW) documents are beautifully typeset.
%\acs{SW} 	SW documents are beautifully typeset.
%\acl{SW} 	Scientific Word documents are beautifully typeset.

\pdfbookmark[0]{\listadesiglasname}{los}
\chapter*{\listadesiglasname}%

% Incluir no sumário
% \addcontentsline{toc}{chapter}{\listadesiglasname}%

\begin{acronym}
  \acro{COSI}{Colegiado do Curso de Sistemas de Informação}
  \acro{DECSI}{Departamento de Computação e Sistemas}
  \acro{ICEA}{Instituto de Ciências Exatas e Aplicadas}
  \acro{JM}{João Monlevade}
  \acro{SI}{Sistemas de Informação}
  \acro{UFOP}{Universidade Federal de Ouro Preto}
	\acro{UML}{\textit{Unified Modeling Language}}
  \acro{CMS}{\textit{Content Managment System}}
  \acro{HTML}{\textit{Hypertext Markup Language}}
  \acro{HTTPS}{\textit{Hypertext Transfer Protocol Secure}}
  \acro{CSS}{\textit{Cascading Style Sheets}}
  \acro{SPA}{\textit{Single Page Aplication}}
  \acro{SEO}{\textit{Search Engine Optimization}}
  \acro{XSS}{\textit{Cross-Site Scripting}}
  \acro{CSRF}{\textit{Cross-Site Request Forgery}}
  \acro{WCAG}{\textit{Web Content Accessibility Guidelines}}
  \acro{DDoS}{\textit{Distributed Denial of Service}}
  \acro{2FA}{\textit{Two Factor Authetication}}
  \acro{MFA}{\textit{Multi Factor Authetication}}
  \acro{SQL}{\textit{Structured Query Language}}
\end{acronym}

% ---

% ---
% inserir lista de símbolos, caso se aplique.
% \include{./pre-textuais/listas/simbolos}
% ---

% ---
% inserir o sumario
% ---
\pdfbookmark[0]{\contentsname}{toc}
\tableofcontents*
\cleardoublepage{}
% ---

% ----------------------------------------------------------
% ELEMENTOS TEXTUAIS
% ----------------------------------------------------------
\textual{}


% ----------------------------------------------------------
% Capítulos
% ----------------------------------------------------------

% ---
% Introdução
% ---
% ----------------------------------------------------------
% Introdução
% ----------------------------------------------------------
\chapter{Introdução}
\label{cap:introducao}
% ----------------------------------------------------------

No mundo contemporâneo, marcado pela conectividade e pelos avanços tecnológicos, a cultura e as artes encontram na tecnologia uma aliada poderosa para sua promoção e difusão. A Casa de Cultura de Monlevade, desempenhando o importante papel de Secretaria Municipal de Cultura na cidade de João Monlevade, tem buscado formas inovadoras de fomentar o talento artístico local. Nesse contexto, o desenvolvimento de uma plataforma de artistas monlevadenses surge como uma iniciativa que visa valorizar e promover a riqueza cultural da região.

No entanto, a criação de uma plataforma cultural esbarra em um desafio recorrente enfrentado por muitas iniciativas culturais no Brasil: a escassez de recursos financeiros. Em um cenário de cortes orçamentários e falta de investimentos, torna-se ainda mais desafiador obter os recursos necessários para viabilizar projetos como esse. É nesse contexto que se busca por alternativas criativas e parcerias gratuitas para tornar possível a concretização da plataforma de artistas monlevadenses.

Além de servir como um meio de divulgação, a plataforma se propõe a ser uma ponte entre os artistas locais e o público, facilitando o acesso à arte produzida na região e permitindo que os artistas ganhem maior visibilidade. Em uma cidade como João Monlevade, cuja cena cultural é rica, mas pouco explorada fora de seus limites, uma ferramenta como essa tem o potencial de fortalecer o setor artístico e criar novas oportunidades de valorização e reconhecimento. Ao reunir informações sobre os artistas, seus portfólios e produções, a plataforma oferece um espaço centralizado para que o público possa conhecer e se conectar com o que é produzido na cidade.

No entanto, a implementação de uma plataforma dessa magnitude enfrenta desafios que vão além da simples escassez de recursos. Há também a necessidade de garantir que ela seja funcional, acessível e atrativa tanto para os artistas quanto para o público. A usabilidade e a intuitividade do sistema são aspectos cruciais para que o projeto seja bem-sucedido, considerando que muitos dos artistas locais podem não ter familiaridade com tecnologias digitais complexas. Portanto, o desenvolvimento de uma interface amigável, que promova a participação ativa dos usuários, é essencial para o engajamento.

Paralelamente aos desafios, a criação dessa plataforma também traz consigo diversas oportunidades. A iniciativa pode fortalecer a identidade cultural da cidade, promovendo uma rede de colaboração entre artistas, instituições culturais e o público. Além disso, a plataforma tem o potencial de fomentar o turismo cultural na cidade, atraindo visitantes interessados em conhecer o trabalho dos artistas monlevadenses e participar dos eventos promovidos pela Casa de Cultura.

Dessa forma, o desenvolvimento da plataforma se apresenta não apenas como uma solução tecnológica, mas também como uma estratégia para a valorização da cultura local, possibilitando que os artistas tenham maior autonomia na divulgação de suas obras e que o público tenha acesso facilitado à produção cultural de João Monlevade.

% ----------------------------
\section{O problema de pesquisa}
\label{sec:problema}

O desenvolvimento de uma plataforma de artistas monlevadenses para a Casa de Cultura esbarra na dificuldade em obter recursos financeiros para sua realização. Em um contexto de restrições orçamentárias e escassez de investimentos na cultura, torna-se desafiador viabilizar a criação de uma plataforma digital que promova a cultura local e valorize os talentos artísticos de Monlevade. A falta de recursos financeiros adequados e sustentáveis se apresenta como um obstáculo para o desenvolvimento e a manutenção dessa iniciativa cultural, comprometendo seu potencial de impacto e alcance na comunidade.

Compreender os desafios relacionados à captação de recursos e buscar alternativas viáveis para financiar o projeto é fundamental para tornar possível a implementação e a continuidade da plataforma de artistas monlevadenses. É necessário encontrar estratégias e principalmente parcerias que permitam superar as restrições orçamentárias e garantir a sustentabilidade financeira dessa plataforma, a fim de fortalecer a cultura local, promover a divulgação das obras de arte e proporcionar uma experiência enriquecedora para o público.

Um desafio que pode surgir é a diversidade de familiaridade com tecnologia entre os artistas, o que torna importante priorizar a acessibilidade e a usabilidade do site para todos os públicos, sejam esses artistas ou cidadãos comuns. A construção do site deve levar em consideração essa questão e implementar soluções para que as informações sejam passadas de forma integra e de fácil entendimento para todos. A simplicidade e interação de fácil dedução é chave para alcançar esse objetivo.

Um fator que não pode ser subestimado é o engajamento da própria comunidade artística e do público local. Mesmo com uma plataforma bem estruturada e acessível, a adesão ao projeto depende da vontade dos artistas de participar e da disposição do público de interagir com o conteúdo disponível. Isso inclui não apenas a divulgação da plataforma, mas também a criação de estratégias para incentivar a participação ativa dos artistas, garantindo que eles vejam valor em fazer parte da iniciativa. Sem um forte engajamento da comunidade, o impacto da plataforma pode ser limitado.

Além disso, um grande problema que é recorrente quando se trata de sistemas em geral é a manutenibilidade do código. O site e o sistema de gestão deve ser desenvolvido não apenas pensando na interação site-usuário, mas também ser bem estruturado na parte do backend para que não seja abandonado pela sua complexidade. A escalabilidade de um sistema é de grande importancia para sua longevidade, tornando as boas práticas de programação um ponto chave para o sucesso do projeto como um todo.

A facilidade de atualização do conteúdo também é outro fator importante a ser considerado. A implementação de um sistema \ac{CMS} resolve o problema fornecendo uma interface amigável para a publicação de artigos, edição de páginas e controle de conteúdos do todo o site, além de restrigir o acesso a parte administrativa de forma efetiva com a criação de usuários e definição de papéis \cite{Baker2013}. A escolha do Wagtail CRX \cite{WagtailCRX}, um \textit{fork} do Wagtail \cite{Wagtail}, sistema de gestão de conteúdo baseado em Python, atende perfeitamente as demandas da Casa de Cultura e permite que o sistema de gestão para a Escola de Artes (projeto da Casa de Cultura de João Monlevade), controle de editais para artistas locais e Vitrine de Artistas esteja no mesmo \textit{codebase} .

Considerando a necessidade de manter o conteúdo da plataforma atualizado e relevante, surge o desafio de implementar processos contínuos de curadoria e gerenciamento de conteúdo. A manutenção da qualidade e pertinência das informações requer um esforço constante de coleta e atualização de dados sobre os artistas, seus trabalhos e eventos culturais. Isso demanda tempo, equipe qualificada e um sistema eficiente para facilitar a atualização, o que pode ser dificultado pela falta de recursos humanos e tecnológicos. Além disso, sem um planejamento de longo prazo, a plataforma corre o risco de se tornar obsoleta ou desatualizada, perdendo seu valor como ferramenta de promoção cultural.

Assim, o problema de pesquisa deste trabalho consiste em como desenvolver uma plataforma que valorize os artistas monlevadenses, garantindo o acesso à cultura e promovendo os talentos artísticos presentes em João Monlevade.

\section{Objetivos}
\label{sec:objetivos}

O objetivo geral deste trabalho é desenvolver uma plataforma digital que funcione como uma vitrine para os talentos artísticos de Monlevade, contribuindo para a promoção e valorização da cultura local. Através dessa plataforma, os artistas terão um espaço dedicado para expor e divulgar suas obras, e o público poderá acessar de maneira interativa e acessível o vasto acervo cultural da cidade. Além disso, será implementado um sistema de gestão informatizado para a Escola de Artes, que facilitará o controle de atividades internas e a publicação de editais, promovendo uma maior organização e eficiência no funcionamento da Casa de Cultura.

Os objetivos específicos são:

\begin{itemize}
	\item Desenvolvimento de um site com sistema \ac{CMS} para publicação de artigos: Implementar um \ac{CMS} que permitirá a fácil publicação e atualização de artigos e notícias relacionadas à cultura local. O objetivo é criar um ambiente digital dinâmico e constantemente atualizado, que promova o engajamento do público.

	\item Desenvolvimento de um sistema de gestão para a Escola de Artes: Criar um sistema integrado que auxilie na administração das atividades da Escola de Artes, como o gerenciamento de alunos, turmas, cursos e eventos. Esse sistema deve ser acessível e fácil de usar para os gestores, com o objetivo de otimizar o fluxo de trabalho e melhorar a eficiência operacional da escola.

	\item Criação da Vitrine de Artistas: Desenvolver uma funcionalidade dentro do site da Casa de Cultura que permita aos artistas locais cadastrar suas obras, biografias e informações relevantes, criando um catálogo interativo acessível ao público. O objetivo é garantir uma plataforma inclusiva, onde todos os artistas possam expor seus trabalhos de forma organizada e fácil de navegar, promovendo maior visibilidade para a produção artística de Monlevade.

	\item Implementação do módulo de Editais: Desenvolver um módulo que facilite a criação, gerenciamento e publicação de editais culturais. Esse sistema permitirá a Casa de Cultura gerir os processos seletivos de maneira mais ágil e transparente, com acesso público às informações por meio do site. O objetivo é assegurar que a divulgação de oportunidades culturais seja feita de forma clara e acessível.

	\item Avaliação das funcionalidades implementadas: As funcionalidades desenvolvidas serão avaliadas com base em critérios como usabilidade, acessibilidade e eficiência. Serão realizados testes de interação com os usuários e gestores para garantir que o sistema atenda às necessidades práticas da Casa de Cultura e dos artistas. Espera-se alcançar uma plataforma funcional e eficaz, que otimize o gerenciamento cultural e fortaleça o vínculo entre a Casa de Cultura e a comunidade artística.
\end{itemize}


\section{Metodologia}
\label{sec:metodologia}

Segundo \cite{moresi2003metodologia}, a metodologia de pesquisa pode ser compreendida como os procedimentos empregados na pesquisa, que incluem uma estratégia, os passos práticos e as técnicas específicas utilizadas para a realização da investigação, ou seja, abrange a coleta de dados, a escolha da amostra, os instrumentos de pesquisa e outras atividades práticas. Neste trabalho, a análise será feita com base no processo de desenvolvimento e implementação da plataforma, avaliando a sua adequação em termos de acessibilidade, usabilidade e impacto positivo para os artistas monlevadenses. Para isso, serão utilizadas ferramentas online, como Page Speed Insights, W3C Validator, GTMetrix, Pentest Tools e ImmuniWeb a fim de garantir o desempenho, a conformidade com os padrões web e a segurança do sistema, dessa forma atendendo às necessidades do público-alvo e promovendo a valorização dos artistas locais.

Os passos para execução deste trabalho são assim definidos:

\begin{itemize}
	\item Fazer uma revisão da literatura sobre os tópicos: \ac{UML} como definido por \citeonline{booch1999uml}, tecnologias empregadas no desenvolvimento do trabalho e pesquisas correlatas ao tema deste estudo;
	\item Desenvolvimento do sistema de gestão para a Casa de Cultura, com os módulos de vitrine de artistas, editais e escola de artes;
	\item Desenvolvimento do sistema \ac{CMS} para publicação de artigos, integrado com o sistema de gestão;
	\item Executar testes no sistema para assegurar que ele funcione corretamente e coletar feedbacks que confirmem sua eficácia.
\end{itemize}

\section{Organização do trabalho}

Neste capítulo, foi definido o contexto do tema tratado neste trabalho, delineando o problema de pesquisa e os objetivos estabelecidos. No Capítulo 2, é feita uma revisão bibliográfica relevante ao tema. O Capítulo 3 consiste no desenvolvimento do trabalho, detalhando a coleta dos requisitos do sistema e alguns diagramas \ac{UML}. Além disso, este capítulo descreve as interfaces do sistema e o funcionamento de cada tela. No Capítulo 4, é abordada a metodologia utilizada para avaliar o sistema, bem como para mensurar a qualidade do sistema. Os resultados dessa avaliação também são discutidos nesta seção. Por fim, o Capítulo 5 apresenta as conclusões do projeto.


% ----------------------------------------------------------
% PARTE
% ----------------------------------------------------------
%\part{Referenciais teóricos}
% ----------------------------------------------------------
% ----------------------------------------------------------
% Capitulo de revisão de literatura
% ----------------------------------------------------------
\chapter{Revisão bibliográfica}
\label{cap:revisao}
% ----------------------------------------------------------

\section{Introdução à Casa de Cultura de João Monlevade}
%Confirmar dados com a casa de cultura 
A Casa de Cultura de João Monlevade desempenha um papel importante no desenvolvimento social e na promoção da cultura local. A instituição foi fundada em 1948 por uma iniciativa de um grupo de intelectuais e artistas locais que perceberam a necessidade de um local dedicado à apreciação das expressões culturais monlevadenses. A Casa de Cultura tem sido um local onde as pessoas se reúnem e celebram a cultura desde então, e ela desempenha um papel importante na vida cultural da cidade.

Ao longo dos anos, a Casa de Cultura tem se destacado por sua diversidade de atividades e qualidade. Apresentações musicais, escola de artes,oficinas, dança e literatura estão entre as atividades da instituição. Além de promover a cultura local, essas atividades visam incentivar a produção artística e intelectual da região, promovendo o debate cultural na comunidade.

A Casa de Cultura tem um público diversificado, desde crianças até idosos. Eles sempre estão procurando oferecer programações e atividades que atendam às diferentes faixas etárias e interesses. Além disso, a organização mantém fortes vínculos com as escolas e com os grupos comunitários, promovendo de forma integrada a educação e a cultura.

O significado da Casa de Cultura de João Monlevade vai além da área cultural. A organização ajuda a construir uma consciência histórica e cultural entre os habitantes locais, promovendo a identidade e a memória da cidade. Além disso, a Casa de Cultura contribui para o enriquecimento da vida cultural da cidade, desempenhando um papel importante na formação de novos talentos e na promoção da diversidade cultural.

Atualmente, a Casa de Cultura de João Monlevade está presente em diversas redes sociais, como por exemplo o Instagram e o Facebook. Através de suas páginas nessas redes, os funcionários da Casa de Cultura publicam sobre eventos próximos e futuros organizados por eles, além de divulgar vagas para os cursos da escola de artes, presente dentro da sede. Essa interação mais próxima com a população se tornou cada vez mais importante, tendo em vista a facilidade de comunicação com o publico alvo através dessas ferramentas.

Assim, a Casa de Cultura de João Monlevade é algo mais do que apenas um promotor de eventos culturais. É um símbolo da identidade e da história da cidade, e também um centro de irradiação cultural que melhora a vida dos habitantes e promove a inclusão social por meio da cultura. Seus esforços para preservar e promover a cultura local são inestimáveis, tornando-a um patrimônio cultural e social significativamente valioso para a comunidade monlevadense.


\section{A relevância da internet e dos sites web para a cultura}
Nos dias atuais, não há dúvida de quão importantes são a internet e os sites web para a cultura. Com o uso crescente da internet como meio de comunicação e divulgação cultural, as organizações culturais têm encontrado uma maneira eficaz de atingir um público cada vez mais amplo e diversificado.

Uma das principais vantagens de ter um site web para uma instituição cultural é aumentar sua visibilidade e alcance. Atualmente, sem um grande investimento em publicidade e marketing, uma instituição cultural não poderia alcançar apenas o público local, mas também aqueles de outras regiões e até mesmo de outros países.

Além disso, o público interessado pode acessar informações por meio de um site. Ele pode fornecer informações sobre a história, missão, projetos e atividades da instituição. Ele também pode fornecer informações práticas, como horários de funcionamento, formas de contato e rotas para chegar lá. Isso facilita a interação do público com a instituição e sua participação em suas atividades.

A capacidade de promover eventos e atividades culturais de forma mais eficaz é outra vantagem de ter um site web. Ao divulgar programações, exposições, espetáculos, cursos e workshops no site, é possível alcançar um público maior e garantir que mais pessoas participem das atividades da instituição. Um site online ajuda a fortalecer a cultura da comunidade, aumenta o envolvimento e o orgulho dos habitantes locais ao fornecer informações sobre a cultura local, sua história, tradições e artistas.

Como resultado, a internet e os sites web são hoje uma ferramenta vital para as instituições culturais, permitindo não apenas um maior alcance e visibilidade, mas também uma maior interação e participação do público. Uma instituição cultural desenvolvendo um site web está contribuindo para a promoção da cultura e o fortalecimento da identidade cultural da comunidade.

A digitalização da cultura e a presença online das instituições culturais também permitem a preservação e disseminação do patrimônio cultural de maneira inédita. Arquivos digitais, galerias de imagens, vídeos de performances e gravações de áudio podem ser disponibilizados online, garantindo que um público global tenha acesso a esses recursos. Essa disponibilidade perpetua o conhecimento cultural, tornando-o acessível para futuras gerações, independentemente das barreiras geográficas.

Além da disseminação de informações, os sites web oferecem um espaço para a experimentação e inovação cultural. Plataformas online podem ser utilizadas para criar exposições virtuais, apresentações interativas e outras formas de arte digital. Esses novos formatos não apenas expandem as possibilidades de expressão artística, mas também atraem públicos que talvez não frequentem eventos culturais tradicionais. Dessa forma, a internet se torna um campo fértil para a criação e a reinvenção cultural.

Outro aspecto importante é o papel dos sites web na educação cultural. Instituições culturais podem oferecer cursos online, workshops e palestras, acessíveis a qualquer pessoa com conexão à internet. Esse tipo de oferta educativa amplia o impacto social das instituições culturais, permitindo que mais pessoas tenham acesso a conhecimentos especializados e a oportunidades de desenvolvimento pessoal e profissional.

Finalmente, os sites web permitem uma comunicação bidirecional entre as instituições culturais e seu público. Por meio de comentários, fóruns e redes sociais integradas, os usuários podem interagir diretamente com as instituições, expressar suas opiniões e participar de discussões sobre temas culturais. Essa interação promove um senso de comunidade e pertencimento, além de proporcionar às instituições um feedback valioso para melhorar suas ofertas e serviços.

Em conclusão, a internet e os sites web transformaram a maneira como a cultura é compartilhada, promovida e consumida. Eles oferecem inúmeras vantagens para as instituições culturais, desde o aumento da visibilidade e alcance até a preservação do patrimônio cultural e a inovação artística. Ao desenvolver uma presença online robusta, as instituições culturais não apenas ampliam seu impacto, mas também fortalecem a identidade cultural e a coesão social das comunidades que servem.

\section{Tecnologias para desenvolvimento de sites web}

As principais tecnologias utilizadas para o desenvolvimento web são bem conhecidas. A um nível básico e falando de forma análoga, primeiramente temos o HTML, que seria uma folha de papel onde podemos estruturar nosso site. Para deixa-lo com a nossa cara, modificamos seu CSS, como se estivéssemos colorindo essa folha de papel. E caso seja necessário que essa folha de papel tenha alguma animação, ou uma automação de algum tipo, usamos o JavaScript para dar vida a pagina.

Apesar de ser posivel fazer varias paginas usando somente essas 3 tecnologias, em algum momento ficara faltando algo ou seria muito trabalhoso desenvolver tudo do zero. Com esse problema, desenvolvedores criaram frameworks e bibliotecas para facilitar e agilizar o desenvolvimento web. Dentre eles temos o React, uma biblioteca JavaScript para a construção de interfaces de usuário, especialmente de aplicativos de página única (SPAs) e o Angular, outro framework JavaScript desenvolvido pelo Google para a construção de aplicativos web dinâmicos.

Quando se precisa de um framework ainda mais robusto, que auxilie na criacao de sites no estilo blog, ou que armazenaram dados, noticias ou outras coisas similares, pode-se utilizar sistemas Gerenciadores de Conteúdo (CMS). Esses sistemas facilitam ainda mais a criacao e manutencao de sites com varias funcionalidades, ajudando na utilizacao de banco de dados, criacao de conteudo e ate mesmo na otimizacao para ferramentas de pesquisa (SEO). Dentre os principais atualmente temos o Wordpress, o CMS mais popular, usado para criar e gerenciar sites e blogs com facilidade. Ele oferece uma vasta gama de plugins e temas, facilitando a personalização e a adição de funcionalidades ao site. O Joomla e um CMS de código aberto que permite a criação de sites poderosos e complexos que oferece flexibilidade e uma comunidade ativa, tornando-o uma escolha robusta para sites de médio a grande porte. O Drupal, um CMS altamente flexível e escalável, ideal para sites que requerem uma personalização intensa, esse e conhecido por sua segurança e capacidade de gerenciar grandes volumes de conteúdo.

Assim viu-se indispensável a utilização de um sistema CMS. Para definir qual das tecnologias seria mais adequada, necessita-se atenção a quais serão as funcionalidades desse sistema e qual objetivo ele quer cumprir. Tendo em vista que o site para a Casa de Cultura de João Monlevade teria dois objetivos principais, sendo esses servir como meio para divulgar noticias e editais da Casa de Cultura e ser também um sistema de gestão para a escola de artes, procurou-se um CMS que possibilitasse essa multitarefa.

O CodeRed CMS é um sistema de gerenciamento de conteúdo baseado no framework Django e no CMS Wagtail. Ele foi projetado para facilitar a criação e a gestão de sites empresariais, com foco em flexibilidade, desempenho e facilidade de uso. o Django é um dos frameworks mais robustos e escaláveis para desenvolvimento web, conhecido por sua segurança e capacidade de lidar com grandes volumes de dados e tráfego. Já o Wagtail e Um CMS flexível e poderoso, desenvolvido sobre o Django, que oferece uma interface de administração amigável e recursos avançados de gestão de conteúdo.

Esse sistema possui funcionalidades avançadas para a gestão de conteúdo. A flexibilidade na criação de páginas, permite a criação de páginas personalizadas com componentes reutilizáveis, facilitando a construção de layouts complexos sem a necessidade de códigos complexos ou repetitivos. Auxilia na publicação de notícias e editais, oferecendo ferramentas integradas para a publicação e gerenciamento de conteúdos como notícias e editais, com recursos de agendamento e controle de versões.

Além de uma interface amigável para a criação de novos conteúdos como artigos e editais, o sistema possibilita a utilização de outra interface que poderá ser acessada por administradores do sistema, o django-admin. Esta vem por padrão quando se utiliza o Django e oferece facilidade na modificação de informações, além de acesso direto aos dados guardadas no banco de dados do site, o que será importante para as funcionalidades da Escola de Artes.

A capacidade de integração do CodeRed CMS com outras ferramentas e sistemas também é um fator crucial para sua escolha. A Casa de Cultura de João Monlevade, ao utilizar esse CMS, poderá facilmente incorporar sistemas de pagamento, plataformas de e-mail marketing e outras aplicações essenciais para o gerenciamento das suas atividades culturais e educacionais. Essa integração garante que todas as necessidades operacionais e administrativas possam ser centralizadas em um único sistema, aumentando a eficiência e a eficácia da gestão.

Outra vantagem significativa do CodeRed CMS é a sua robusta estrutura de segurança. Desenvolvido sobre o Django, que é conhecido por suas práticas de segurança integradas, o sistema oferece proteção contra ataques comuns, como SQL injection, cross-site scripting (XSS) e cross-site request forgery (CSRF). Isso é particularmente importante para a Casa de Cultura de João Monlevade, que pode lidar com dados sensíveis dos usuários e transações financeiras.

A escalabilidade do CodeRed CMS também é um ponto forte. Conforme a Casa de Cultura de João Monlevade cresce e expande suas atividades, o CMS pode acompanhar esse crescimento sem comprometer o desempenho. O Django, com seu suporte a grandes volumes de dados e tráfego, assegura que o site permanecerá rápido e responsivo, mesmo com um aumento significativo no número de visitantes e conteúdos publicados.

A facilidade de uso do Wagtail, que é a base do CodeRed CMS, proporciona uma experiência administrativa intuitiva. A interface de arrastar e soltar, juntamente com a capacidade de pré-visualizar as alterações antes de publicá-las, facilita o trabalho dos administradores do site, que podem não ter habilidades técnicas avançadas. Isso permite que a equipe da Casa de Cultura de João Monlevade se concentre mais na criação e gestão de conteúdo de qualidade, ao invés de se preocupar com os aspectos técnicos da manutenção do site.

Além disso, o CodeRed CMS suporta práticas de SEO avançadas, essenciais para aumentar a visibilidade online da Casa de Cultura de João Monlevade. Ferramentas integradas ajudam a otimizar o conteúdo para motores de busca, melhorando o ranking do site nos resultados de pesquisa e atraindo mais visitantes. Com uma estratégia de SEO bem implementada, a Casa de Cultura pode alcançar um público mais amplo, promovendo suas atividades e eventos de maneira mais eficaz.

A escolha do CodeRed CMS também se justifica pela possibilidade de personalização e extensão das suas funcionalidades. A Casa de Cultura de João Monlevade pode, por exemplo, desenvolver módulos específicos para atender a necessidades particulares, como a gestão de inscrições para cursos e workshops, acompanhamento de presença dos alunos, e avaliação de performances artísticas. Essa capacidade de personalização garante que o sistema possa evoluir junto com as demandas específicas da instituição.

A manutenção e atualização do site também são facilitadas pelo uso do CodeRed CMS. A comunidade ativa de desenvolvedores de Django e Wagtail oferece suporte contínuo, documentação extensa e atualizações regulares, o que garante que o sistema esteja sempre atualizado com as últimas inovações tecnológicas e práticas de segurança. Isso reduz o risco de vulnerabilidades e garante que o site opere de maneira eficiente e segura.

Outro aspecto importante é a capacidade do CodeRed CMS de fornecer análises detalhadas sobre o comportamento dos usuários. Integrado com ferramentas de análise como Google Analytics, o sistema permite que a Casa de Cultura de João Monlevade monitore o tráfego do site, entenda as preferências dos visitantes e tome decisões informadas sobre o conteúdo e as funcionalidades a serem aprimoradas. Esse conhecimento detalhado do público-alvo é crucial para a criação de estratégias de engajamento mais eficazes e para a otimização da experiência do usuário.

Além disso, a integração do CodeRed CMS com plataformas de mídias sociais facilita a promoção das atividades da Casa de Cultura. Compartilhar eventos, exposições e outras atividades diretamente nas redes sociais ajuda a aumentar a visibilidade e a atrair um público maior. As funcionalidades de compartilhamento social integradas garantem que os conteúdos possam ser facilmente divulgados, aumentando o alcance e o impacto das campanhas de marketing digital.

A utilização do CodeRed CMS também contribui para a sustentabilidade da Casa de Cultura de João Monlevade. Um sistema eficiente de gerenciamento de conteúdo reduz a necessidade de recursos físicos, como papel para folhetos e cartazes, substituindo-os por versões digitais acessíveis online. Isso não apenas diminui os custos operacionais, mas também apoia práticas ambientalmente responsáveis.

A interface do usuário do Wagtail, intuitiva e amigável, facilita a capacitação da equipe da Casa de Cultura. Com um treinamento básico, os administradores do site podem rapidamente aprender a criar, editar e publicar conteúdos, garantindo que a plataforma seja utilizada de maneira eficiente. A facilidade de uso incentiva a equipe a atualizar regularmente o site, mantendo-o sempre relevante e atualizado para os visitantes.

Em termos de design, o CodeRed CMS permite a implementação de um site visualmente atraente e alinhado com a identidade visual da Casa de Cultura de João Monlevade. A flexibilidade na personalização do layout e a disponibilidade de templates pré-desenvolvidos proporcionam um design profissional sem a necessidade de investimentos significativos em desenvolvimento gráfico. Isso é fundamental para transmitir a imagem e os valores da instituição de maneira coerente e atraente.

Por fim, o suporte a múltiplos idiomas é uma funcionalidade relevante do CodeRed CMS. Considerando a diversidade cultural e o potencial interesse de visitantes internacionais, a possibilidade de oferecer conteúdo em diferentes idiomas amplia o alcance do site e facilita o acesso de um público global. Essa funcionalidade é particularmente importante para a promoção de eventos e exposições que possam atrair turistas e estudiosos de outros países.

Em conclusão, a escolha do CodeRed CMS para o desenvolvimento do site da Casa de Cultura de João Monlevade é fundamentada em uma série de benefícios que incluem flexibilidade, segurança, escalabilidade, facilidade de uso, capacidade de personalização e suporte a práticas de SEO e marketing digital. Essas características garantem que o site não apenas atenda às necessidades imediatas da instituição, mas também esteja preparado para evoluir e crescer junto com ela, promovendo a cultura e o desenvolvimento social da comunidade de maneira eficaz e sustentável.

\section{Usabilidade e acessibilidade de sites web}

A usabilidade e a acessibilidade são elementos cruciais para garantir que um site seja fácil de usar e acessível a todos os públicos, incluindo pessoas com deficiência. A usabilidade refere-se à eficiência, eficácia e satisfação com que os usuários podem realizar tarefas em um site, enquanto a acessibilidade diz respeito à capacidade do site de ser utilizado por pessoas com diversas deficiências. Para criar um site verdadeiramente inclusivo, é essencial seguir diretrizes e princípios que promovam tanto a usabilidade quanto a acessibilidade.

Primeiramente, a simplicidade e clareza da interface são fundamentais para a usabilidade. Uma interface bem projetada deve ser intuitiva e fácil de entender, mesmo para usuários que a estão acessando pela primeira vez. Isso inclui a organização lógica dos elementos, a utilização de uma linguagem clara e a minimização da complexidade. A adoção de uma abordagem minimalista, onde apenas os elementos essenciais são destacados, pode ajudar a evitar a sobrecarga de informações e tornar a navegação mais fluida.

Além da simplicidade, a navegação intuitiva é um aspecto vital para a usabilidade. Os menus e as opções de navegação devem ser organizados de forma coerente, permitindo que os usuários encontrem facilmente o que procuram. Elementos de navegação como barras de menu, botões e links devem ser visíveis e consistentes em todas as páginas do site. Adotar padrões conhecidos e evitar mudanças drásticas na estrutura de navegação pode melhorar significativamente a experiência do usuário.

A acessibilidade vai além da usabilidade, garantindo que o conteúdo do site esteja disponível e utilizável por pessoas com deficiências visuais, auditivas e motoras. Para alcançar esse objetivo, é necessário seguir diretrizes específicas, como as estabelecidas pela Web Content Accessibility Guidelines (WCAG). Essas diretrizes fornecem recomendações detalhadas para tornar o conteúdo web mais acessível.

Para pessoas com deficiências visuais, é importante que o site seja compatível com leitores de tela. Isso inclui o uso adequado de tags alt para descrever imagens, a estruturação correta do HTML para que os leitores de tela possam interpretar a página de maneira lógica e a garantia de contraste adequado entre o texto e o fundo para facilitar a leitura. Além disso, oferecer opções de aumentar o tamanho da fonte pode ser benéfico para usuários com baixa visão.

Para tornar o conteúdo acessível a pessoas com deficiências auditivas, é essencial fornecer alternativas textuais para qualquer conteúdo de áudio ou vídeo. Legendas, transcrições e descrições de áudio são recursos importantes que garantem que essas informações possam ser acessadas por todos. Em vídeos, as legendas sincronizadas e as transcrições ajudam a transmitir a mensagem de maneira eficaz.

A acessibilidade para pessoas com deficiências motoras pode ser melhorada garantindo que o site possa ser navegado usando apenas o teclado, sem a necessidade de um mouse. Os elementos clicáveis devem ser suficientemente grandes e espaçados para facilitar o uso por pessoas com mobilidade limitada. Além disso, implementar atalhos de teclado pode acelerar a navegação para usuários que dependem desses dispositivos.

Para garantir que a Casa de Cultura de João Monlevade ofereça um site inclusivo, é vital incorporar as melhores práticas de usabilidade e acessibilidade desde o início do processo de desenvolvimento. Uma abordagem proativa à acessibilidade não apenas beneficia os usuários com deficiências, mas também melhora a experiência geral para todos os visitantes do site.

Um dos primeiros passos para alcançar uma boa acessibilidade é realizar testes com usuários reais, incluindo pessoas com diferentes tipos de deficiência. Esses testes fornecem insights valiosos sobre os desafios que esses usuários enfrentam e ajudam a identificar áreas de melhoria que podem não ser evidentes para desenvolvedores sem essas experiências. Feedback direto dos usuários é uma ferramenta poderosa para refinar a usabilidade e garantir que as soluções implementadas realmente atendam às suas necessidades.

Além disso, a implementação de padrões e ferramentas de acessibilidade deve ser uma prática contínua. Ferramentas como validadores de acessibilidade e verificadores de contraste de cores podem ajudar a identificar e corrigir problemas de acessibilidade durante o desenvolvimento. Manter-se atualizado com as diretrizes WCAG e outras normas de acessibilidade garante que o site continue a ser acessível à medida que novas tecnologias e padrões emergem.

O treinamento da equipe da Casa de Cultura de João Monlevade em princípios de usabilidade e acessibilidade é igualmente importante. Capacitar os administradores e criadores de conteúdo para entender e aplicar essas práticas assegura que o site permaneça acessível a longo prazo. Treinamentos regulares e recursos educacionais sobre acessibilidade web podem ajudar a manter a equipe informada sobre as melhores práticas e novos desenvolvimentos na área.

Além dos aspectos técnicos, a acessibilidade também envolve considerar as necessidades culturais e linguísticas dos usuários. Para uma instituição cultural como a Casa de Cultura de João Monlevade, é essencial que o site reflita a diversidade e a riqueza cultural da comunidade. Oferecer conteúdo em múltiplos idiomas e garantir que as traduções sejam precisas e culturalmente relevantes é uma forma de tornar o site mais inclusivo e acessível para um público mais amplo.

Outro elemento crucial é a personalização da experiência do usuário. Permitir que os visitantes ajustem as configurações do site de acordo com suas preferências individuais, como tamanhos de fonte, esquemas de cores e opções de contraste, pode melhorar significativamente a usabilidade e a acessibilidade. Funcionalidades como modos de alto contraste e leitores de texto integrados são exemplos de como a personalização pode atender a diversas necessidades de acessibilidade.

Para promover a inclusão digital, é importante também considerar a acessibilidade em dispositivos móveis. Com o crescente uso de smartphones e tablets para acessar a internet, o site da Casa de Cultura de João Monlevade deve ser responsivo e oferecer uma experiência de usuário consistente em todas as plataformas. Garantir que o design seja mobile-first, ou seja, otimizado primeiro para dispositivos móveis e depois adaptado para desktops, pode ser uma estratégia eficaz para alcançar essa meta.

Assim, é essencial que a acessibilidade e a usabilidade sejam vistas como uma responsabilidade contínua e colaborativa. Manter um diálogo aberto com a comunidade de usuários, incluindo aqueles com deficiências, e estar disposto a adaptar e melhorar o site com base em suas necessidades e feedbacks, é a chave para criar um ambiente digital verdadeiramente inclusivo. A construção do site, ao adotar uma abordagem centrada no usuário para a acessibilidade, não apenas cumpre suas obrigações legais e éticas, mas também reforça seu compromisso com a inclusão e a equidade.


\section{Segurança da informação em sites web}

A proteção da informação em sites da web é uma questão crucial, especialmente para instituições como a Casa de Cultura de João Monlevade, que lida com dados sensíveis de usuários e informações essenciais para suas atividades. Os principais desafios em termos de segurança incluem as ameaças de hackers, malware e phishing, todos capazes de causar danos significativos, desde a interrupção dos serviços até o comprometimento dos dados pessoais dos usuários.

Os ataques cibernéticos representam uma das maiores ameaças à segurança dos sites da web. Esses ataques podem se manifestar de várias formas, incluindo invasões para roubo de informações, defacement (alteração maliciosa do conteúdo do site) e ataques de negação de serviço (DDoS), que podem tornar o site inacessível aos usuários. Para mitigar esses riscos, é vital adotar medidas robustas de segurança, como firewalls, sistemas de detecção de intrusos e a configuração adequada dos servidores.

Outra ameaça relevante é o malware, software malicioso criado para danificar, explorar ou desativar sistemas computacionais. O malware pode ser introduzido no site por diversas vias, incluindo downloads de arquivos infectados, vulnerabilidades em plugins e temas e até mesmo via anúncios maliciosos. Para manter o site seguro contra malwares, é fundamental manter o software sempre atualizado, fazer verificações de segurança frequentes e garantir que todos os plugins e temas sejam baixados de fontes confiáveis e atualizados regularmente.

O phishing é uma técnica de engenharia social usada para enganar os usuários e obter informações confidenciais, como nomes de usuário, senhas e detalhes de cartão de crédito. Sites fraudulentos que imitam o site legítimo da Casa de Cultura de João Monlevade podem ser usados para enganar os usuários. Para proteger contra phishing, é importante educar os usuários sobre os riscos e implementar medidas de autenticação robustas, como a autenticação de dois fatores (2FA).

Para garantir a segurança do site da Casa de Cultura de João Monlevade, a implementação de protocolos de segurança é essencial. Isso inclui o uso de HTTPS para criptografar a comunicação entre o navegador do usuário e o servidor, protegendo assim as informações transmitidas contra interceptação. Além disso, a implementação de certificados SSL/TLS é crucial para assegurar que os dados trocados permaneçam confidenciais e íntegros.

A atualização constante do software é outra medida crítica para a segurança do site. Manter o sistema operacional, servidores, CMS e todos os plugins e temas atualizados é vital para proteger contra vulnerabilidades conhecidas que podem ser exploradas por hackers. As atualizações de software frequentemente incluem patches de segurança que corrigem falhas recentemente descobertas, tornando o site mais seguro contra ataques.

A proteção contra ataques cibernéticos também envolve a realização de backups regulares dos dados do site. Ter backups atualizados garante que, em caso de um ataque bem-sucedido ou falha do sistema, o site possa ser restaurado rapidamente com a perda mínima de dados. Além disso, é recomendável implementar soluções de monitoramento e resposta a incidentes, que podem detectar atividades suspeitas e responder rapidamente para mitigar os danos.

Além das medidas mencionadas, é vital realizar auditorias regulares de segurança. Essas auditorias ajudam a identificar possíveis vulnerabilidades antes que sejam exploradas por atacantes. A realização de testes de penetração (pentests) é uma prática eficaz para avaliar a segurança do site, simulando ataques reais e identificando pontos fracos que necessitam de correção. A Casa de Cultura de João Monlevade deve considerar contratar profissionais de segurança cibernética para realizar essas avaliações periódicas.

A conscientização e o treinamento dos funcionários também desempenham um papel crucial na segurança da informação. Os funcionários devem ser treinados para reconhecer e responder a ameaças cibernéticas, como e-mails de phishing e outras tentativas de engenharia social. Programas de treinamento contínuo garantem que todos na organização estejam atualizados sobre as melhores práticas de segurança e saibam como proteger as informações sensíveis do site e dos usuários.

Outro aspecto importante é a gestão de acessos e permissões. O princípio do menor privilégio deve ser aplicado, garantindo que cada usuário tenha apenas o acesso necessário para realizar suas tarefas. Contas com privilégios administrativos devem ser limitadas e monitoradas rigorosamente para evitar abuso de acesso. Além disso, é importante desativar ou remover contas de usuários que não são mais necessárias para minimizar o risco de uso indevido.

A implementação de autenticação multifator (MFA) adiciona uma camada extra de segurança, exigindo que os usuários forneçam duas ou mais formas de verificação antes de acessar suas contas. Isso pode incluir algo que eles conhecem (como uma senha), algo que possuem (como um token ou smartphone) e algo que são (como biometria). A MFA reduz significativamente o risco de acesso não autorizado, mesmo que uma senha seja comprometida.

A política de gerenciamento de incidentes deve estar bem estabelecida. Isso inclui a definição de procedimentos claros para responder a incidentes de segurança, a comunicação interna e externa durante um incidente, e a recuperação e análise pós-incidente. Ter um plano de resposta a incidentes bem estruturado permite que a organização responda de maneira eficiente e minimize os impactos de quaisquer ataques cibernéticos.

Finalmente, a colaboração com outras organizações e a participação em comunidades de segurança cibernética pode fornecer insights valiosos e atualizações sobre as últimas ameaças e tendências de segurança. A troca de informações e melhores práticas com outras instituições culturais pode fortalecer as defesas cibernéticas da Casa de Cultura de João Monlevade.

A segurança da informação é uma preocupação contínua e multifacetada que requer uma abordagem proativa e abrangente. Para a Casa de Cultura de João Monlevade, a proteção dos dados e a manutenção da confiança dos usuários são essenciais para o sucesso de sua plataforma digital. Ao implementar medidas robustas de segurança, realizar auditorias regulares, treinar funcionários, gerenciar acessos de forma rigorosa e adotar tecnologias avançadas como a autenticação multifator, a instituição pode criar um ambiente seguro e confiável para todos os seus usuários.

Estas práticas não apenas protegem contra as ameaças cibernéticas, mas também garantem a continuidade das operações e a integridade das informações. A segurança deve ser vista como um componente essencial do desenvolvimento e manutenção do site, integrando-se a todas as fases do ciclo de vida do software. Através de um compromisso contínuo com a segurança da informação, a Casa de Cultura de João Monlevade pode cumprir sua missão de promover a cultura e servir a comunidade de maneira segura e eficiente.

\section{Trabalhos Relacionados}

A análise de trabalhos relacionados é fundamental para o desenvolvimento de um projeto de site para a Casa de Cultura de João Monlevade. Este capítulo aborda exemplos de sites web de Casas de Cultura no Brasil e no mundo, focando em suas funcionalidades, design e usabilidade, bem como na qualidade do conteúdo e informação disponibilizada. A identificação de boas práticas e tendências emergentes oferece uma base sólida para o desenvolvimento de um site que atenda às necessidades da comunidade e promova a cultura de maneira eficaz.

Para compreender melhor as melhores práticas na criação de sites para Casas de Cultura, foram analisados vários exemplos, tanto nacionais quanto internacionais. A seleção incluiu instituições com características semelhantes à Casa de Cultura de João Monlevade, permitindo uma comparação direta e a extração de elementos aplicáveis ao contexto local.

A Casa de Cultura Laura Alvim, localizada no Rio de Janeiro, possui um site que se destaca pela clareza e acessibilidade das informações. Entre suas funcionalidades, incluem-se a divulgação de eventos, exposições e cursos, além de uma área dedicada à programação cultural e à história da instituição. O design é responsivo, adaptando-se bem a dispositivos móveis, e utiliza uma paleta de cores que facilita a navegação. A usabilidade é aprimorada por menus claros e uma estrutura de navegação intuitiva. A qualidade do conteúdo é alta, com informações detalhadas sobre as atividades e serviços oferecidos.

O Southbank Centre, em Londres, é um dos maiores centros culturais da Europa. Seu site é um exemplo de como integrar diversas funcionalidades de maneira coesa. Oferece uma plataforma robusta para a venda de ingressos, uma agenda detalhada de eventos e espaços dedicados a exposições permanentes e temporárias. O design é moderno e visualmente atraente, com uma interface amigável que prioriza a experiência do usuário. A qualidade da informação é rigorosamente mantida, com descrições precisas e atraentes que incentivam a participação do público.

O Centro Cultural São Paulo é um exemplo de como um site pode ser utilizado para engajar a comunidade. Além de informações sobre eventos e exposições, o site inclui um blog com artigos sobre cultura e arte, promovendo um diálogo contínuo com o público. A funcionalidade de reserva de espaços e inscrição em cursos diretamente pelo site facilita o acesso aos serviços oferecidos. O design é limpo e funcional, focando na simplicidade para garantir a fácil navegação. A qualidade da informação é consistente, com atualizações frequentes e conteúdo relevante para os visitantes.

A partir da análise dos exemplos acima, diversas boas práticas e tendências foram identificadas, servindo como diretrizes para o desenvolvimento do site da Casa de Cultura de João Monlevade. As funcionalidades devem ir além da simples apresentação de informações, proporcionando interatividade e facilidades aos usuários. A inclusão de calendários de eventos, áreas dedicadas a artistas locais e blogs são exemplos de como engajar o público e facilitar o acesso aos serviços culturais.

O design do site deve ser responsivo e adaptável a diversos dispositivos, garantindo uma boa experiência tanto em desktops quanto em dispositivos móveis. A usabilidade deve ser uma prioridade, com menus intuitivos, navegação fácil e informações acessíveis. A adoção de padrões de design que facilitem a leitura e a interação é essencial para manter os usuários engajados. A qualidade da informação deve ser mantida com conteúdos detalhados, atualizados e relevantes. Descrições precisas de eventos, perfis de artistas, artigos sobre cultura e arte, e notícias sobre a instituição ajudam a criar um site rico e informativo. Além disso, a linguagem utilizada deve ser clara e acessível, atingindo um público diversificado.


% ----------------------------------------------------------
% PARTE
% ----------------------------------------------------------
%\part{Desenvolvimento}
% ----------------------------------------------------------
% ----------------------------------------------------------
% Desenvolvimento
% ----------------------------------------------------------
\chapter{Desenvolvimento}
\label{cap:desenvolvimento}
% ----------------------------------------------------------



% ----------------------------------------------------------
% PARTE
% ----------------------------------------------------------
%\part{Resultados}
% ----------------------------------------------------------
% ----------------------------------------------------------
% Capítulo de Resultados
% ----------------------------------------------------------
\chapter{Resultados}
\label{cap:resultados}
% ---

Em termos de gestão institucional, a plataforma trouxe uma modernização considerável para a Casa de Cultura, ao automatizar processos importantes como a gestão de editais e o acompanhamento da Escola de Artes. Com a seção de editais, foi possível integrar um sistema de inscrição online que facilita o processo de seleção e organização de eventos e concursos culturais, que antes era feita através de formulários do Google. Isso gera economia de tempo e recursos para os administradores e aumenta a transparência nas seleções, já que todas as informações ficam disponíveis de forma clara e acessível. Da mesma forma, a Escola de Artes se beneficiou de diversas funcionalidades que permitem o acompanhamento mais próximo dos alunos, com a informatização dos documentos acadêmicos, facilitando o acesso a informação entre alunos, professores e responsáveis.

Os testes realizados também reforçaram o bom desempenho da plataforma. Durante os testes de usabilidade, foi possível identificar que a navegação pela plataforma é intuitiva, com poucos ajustes necessários. Os usuários reportaram uma experiência positiva, tanto em termos de usabilidade quanto de layout visual, destacando a facilidade de localizar as principais funcionalidades e a clareza na exibição das informações.

Serão apresentados os resultados obtidos durante o processo de testes e validação da plataforma de artistas monlevadenses. Os testes foram realizados com base nos critérios de avaliação definidos previamente: funcionalidade, usabilidade, desempenho e segurança. A seguir, detalhamos os métodos aplicados para cada critério e os resultados observados.

\section{Critérios de Avaliação}

Funcionalidade: O principal objetivo foi verificar se as funcionalidades planejadas, como publicação de notícias, exposição dos artistas na "Vitrine", acompanhamento da Escola de Artes e divulgação de editais, estavam funcionando corretamente conforme o escopo do projeto.

Usabilidade: Avaliamos a facilidade de uso da plataforma, levando em consideração a experiência do usuário, navegabilidade e acessibilidade. Foi importante assegurar que a interface da plataforma fosse intuitiva, proporcionando uma experiência eficiente tanto para os artistas quanto para os administradores do sistema.

Desempenho: Medimos o tempo de resposta da plataforma sob diferentes cenários de uso, simulando múltiplos acessos simultâneos e a consulta de grandes volumes de dados. Verificamos a capacidade do sistema em lidar com essas demandas sem prejuízo à experiência do usuário.

Segurança: Avaliamos a integridade e a proteção dos dados inseridos na plataforma, garantindo que as informações dos usuários, especialmente dos artistas e administradores, estivessem protegidas contra acessos não autorizados.

\section{Métodos de Avaliação}
Testes de Usabilidade: Foram realizados testes com um grupo de usuários da Casa de Cultura, incluindo artistas e administradores. Esses testes tiveram como foco observar a interação dos usuários com as diferentes seções da plataforma, como o formulário de inscrição de artistas e o painel administrativo. Através de ferramentas de gravação de sessão e feedback direto, foram identificados pontos de melhoria, como o reposicionamento de alguns botões e a simplificação de etapas no processo de cadastro.

Testes de Desempenho: Utilizamos ferramentas automatizadas de teste de carga, simulando o acesso simultâneo de diversos usuários ao sistema, especialmente em áreas que demandam maior processamento, como a página de "Editais" e a "Vitrine de Artistas". O tempo médio de resposta do servidor foi inferior a 2 segundos, o que demonstra um desempenho eficiente para o volume esperado de acessos. No entanto, em testes de carga extrema, foi identificado um leve aumento no tempo de resposta, apontando áreas potenciais para otimização.

Análise de Logs: Foram analisados os logs gerados pela plataforma durante o período de testes, com o intuito de identificar possíveis erros ou comportamentos anômalos. Através dessa análise, foi possível verificar a consistência das operações de banco de dados, bem como a integridade das transações realizadas pelos administradores e usuários da plataforma. Pequenos erros de validação de dados foram corrigidos antes da finalização dos testes.

\section{Resultados Obtidos}
Os resultados dos testes indicam que a plataforma atende aos critérios de funcionalidade e usabilidade estabelecidos no início do projeto. As funcionalidades principais operam corretamente, e a interface foi bem avaliada pelos usuários quanto à sua facilidade de uso. Com relação ao desempenho, o sistema mostrou-se robusto, respondendo adequadamente à carga de usuários simulada, sendo necessário apenas pequenos ajustes para otimizar o tempo de resposta em cenários de alta demanda.

Quanto à segurança, os testes não revelaram vulnerabilidades críticas, e as medidas de proteção dos dados se mostraram eficazes. Foram implementadas rotinas de backup automáticas e mecanismos de autenticação robustos para garantir a integridade das informações.

Em resumo, a plataforma da Casa de Cultura de João Monlevade se mostrou eficiente nos testes realizados, com bom desempenho, usabilidade adequada e segurança confiável, estando apta a ser utilizada pelos artistas locais e administradores da instituição.

% ----------------------------------------------------------
% Finaliza a parte no bookmark do PDF
% para que se inicie o bookmark na raiz
% e adiciona espaço de parte no Sumário
% ----------------------------------------------------------
\phantompart{}

% ----------------------------------------------------------
% Conclusão
% ----------------------------------------------------------
% ----------------------------------------------------------
% Conclusão
% ----------------------------------------------------------
\chapter[Conclusão]{Conclusão}
%\addcontentsline{toc}{chapter}{Conclusão}
% ---

O desenvolvimento da plataforma de artistas monlevadenses para a Casa de Cultura de João Monlevade resultou em uma solução robusta e inovadora que atende às necessidades da instituição e dos artistas locais. A proposta inicial foi criar um sistema que reunisse, em um só ambiente, a divulgação de notícias culturais, a gestão de editais, o acompanhamento das atividades da Escola de Artes e, principalmente, a criação de uma vitrine digital para os artistas da cidade poderem expor seus trabalhos. Esses objetivos foram plenamente atingidos e validados durante os testes, o que demonstrou a funcionalidade e eficácia da plataforma para essas finalidades.

A relevância do site para a Casa de Cultura de João Monlevade se evidencia em vários aspectos. A vitrine digital dos artistas, por exemplo, representa um avanço significativo para a promoção da cultura local. Artistas de diferentes áreas — como música, teatro, artes plásticas e dança — agora possuem um espaço online dedicado à exposição de seus trabalhos. Esse ambiente proporciona uma maior visibilidade para os artistas da cidade, especialmente aqueles que ainda não possuem um portfólio digital consolidado, democratizando o acesso à divulgação de arte e cultura. Além disso, a inclusão de funcionalidades que permitem a inserção de links, vídeos e imagens ampliou as possibilidades de exposição, tornando a plataforma mais interativa e visualmente atrativa tanto para artistas quanto para o público.

Apesar dos resultados positivos, o desenvolvimento da plataforma abre portas para futuras pesquisas e melhorias. Entre as sugestões para trabalhos futuros, está a possibilidade de integrar novas tecnologias que possam aprimorar a experiência do usuário. Um exemplo disso seria a utilização de algoritmos de recomendação baseados em aprendizado de máquina, que sugeririam artistas ou eventos culturais de acordo com os interesses do usuário. Tal funcionalidade poderia aumentar o engajamento do público com a plataforma, personalizando a experiência de navegação e promovendo uma maior interação entre os artistas e o público.

Outra área que merece destaque para pesquisas futuras é a ampliação da acessibilidade da plataforma. Embora o projeto atual já siga boas práticas de usabilidade, como a organização clara de informações e o design responsivo, há espaço para explorar soluções mais inclusivas. Isso inclui a implementação de tecnologias assistivas que garantam o acesso de pessoas com deficiências, como leitores de tela para pessoas com deficiência visual e comandos de voz para navegação. Dessa forma, a plataforma poderia atender a um público ainda mais amplo, cumprindo seu papel de ser um espaço inclusivo e democrático para a promoção da arte e cultura.

Em termos de expansão de funcionalidades, uma possibilidade interessante seria o desenvolvimento de uma seção de monetização, onde artistas poderiam vender suas obras ou serviços diretamente pela plataforma. Isso poderia ser implementado na forma de um marketplace, permitindo que os artistas locais comercializem seus trabalhos de forma mais acessível e, ao mesmo tempo, gerando uma nova fonte de renda para a Casa de Cultura. Outro aspecto a ser explorado seria o estabelecimento de parcerias com outras instituições culturais, criando eventos colaborativos ou exposições itinerantes que utilizassem a plataforma como um meio de divulgação e gestão.

Por fim, a pesquisa futura poderia também focar no impacto socioeconômico da plataforma. A vitrine digital dos artistas tem o potencial de se transformar em uma ferramenta relevante para o turismo cultural em João Monlevade, atraindo visitantes e apreciadores de arte interessados em conhecer o trabalho dos artistas locais. Estudar os efeitos dessa plataforma no desenvolvimento cultural e econômico da cidade seria um campo de pesquisa interessante, além de fornecer dados importantes para futuras políticas culturais na região.

Dessa forma, o projeto não apenas cumpre seu papel de modernizar as operações da Casa de Cultura, como também oferece uma base sólida para inovações que podem impactar a comunidade artística e cultural de João Monlevade de maneira significativa e duradoura.

% ---

% ----------------------------------------------------------
% ELEMENTOS PÓS-TEXTUAIS
% ----------------------------------------------------------
\postextual%
% ----------------------------------------------------------

% ----------------------------------------------------------
% Referências bibliográficas
% ----------------------------------------------------------
\bibliography{./bib/decsi-cosi-modelo-monografia}

% ----------------------------------------------------------
% Glossário
% ----------------------------------------------------------
%
% Consulte o manual da classe abntex2 para orientações sobre o glossário.
%
%\glossary

% ----------------------------------------------------------
% Apêndices
% ----------------------------------------------------------

% ---
% Inicia os apêndices
% ---
%\begin{apendicesenv}

% Imprime uma página indicando o início dos apêndices
%\partapendices%

% ---
% Primeiro apendice
%% ----------------------------------------------------------
\chapter{Materiais elaborados pelo autor}
\label{cap:apendice}
% ----------------------------------------------------------


% ---

% ---
% Demais apendices
% % ----------------------------------------------------------
\chapter{Materiais elaborados pelo autor}
\label{cap:apendice}
% ----------------------------------------------------------


% ---


%\end{apendicesenv}
% ---

% ----------------------------------------------------------
% Anexos
% ----------------------------------------------------------

% ---
% Inicia os anexos
% ---
%\begin{anexosenv}

% Imprime uma página indicando o início dos anexos
%\partanexos%

% ---
% Primeiro anexo
%% ---
\chapter{Outros materiais}
\label{cap:anexo}
% ---

% ---

% ---
% Capitulo com exemplos de comandos inseridos de arquivo externo
% ---
%\include{./pos-textuais/anexos/abntex2-modelo-include-comandos}
% ---

% ---
% Demais anexos
% % ---
\chapter{Outros materiais}
\label{cap:anexo}
% ---

% ---

%\end{anexosenv}

%---------------------------------------------------------------------
% INDICE REMISSIVO
%---------------------------------------------------------------------
\phantompart%
\printindex
%---------------------------------------------------------------------

\end{document}
